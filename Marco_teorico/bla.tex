%%%%%%%%%%%%%%%%%%%%%%%%%%%%%%%%%%%%%%%%%%%%%%%%%%%%%%%%%%%%%%%
%
% Welcome to Overleaf --- just edit your LaTeX on the left,
% and we'll compile it for you on the right. If you give
% someone the link to this page, they can edit at the same
%
% time. See the help menu above for more info. Enjoy!
% Note: you can export the pdf to see the result at full
% resolution.
%
%%%%%%%%%%%%%%%%%%%%%%%%%%%%%%%%%%%%%%%%%%%%%%%%%%%%%%%%%%%%%%%
\documentclass[12pt,spanish,fleqn,openany,letterpaper,pagesize]{article}

\usepackage{tikz}
\usepackage{verbatim}

\begin{document}

\def\layersep{2.5cm}

\begin{tikzpicture}[shorten >=1pt,->,draw=black!50, node distance=\layersep]
    \tikzstyle{every pin edge}=[<-,shorten <=1pt]
    \tikzstyle{neurona}=[circle,fill=black!25,minimum size=17pt,inner sep=0pt]
    \tikzstyle{neurona de entrada}=[neurona, fill=green!50];
    \tikzstyle{neurona de salida}=[neurona, fill=red!50];
    \tikzstyle{neurona oculta}=[neurona, fill=blue!50];
    \tikzstyle{annot} = [text width=4em, text centered]

    % Draw the input layer nodes
    \foreach \name / \y in {1,...,4}
    % This is the same as writing \foreach \name / \y in {1/1,2/2,3/3,4/4}
        \node[neurona de entrada, pin=left:Entrada \#\y] (I-\name) at (0,-\y) {};

    % Draw the hidden layer nodes
    \foreach \name / \y in {1,...,5}
        \path[yshift=0.5cm]
            node[neurona oculta] (H-\name) at (\layersep,-\y cm) {};

    % Draw the output layer node
    \node[neurona de salida,pin={[pin edge={->}]right:Salida}, right of=H-3] (O) {};

    % Connect every node in the input layer with every node in the
    % hidden layer.
    \foreach \source in {1,...,4}
        \foreach \dest in {1,...,5}
            \path (I-\source) edge (H-\dest);

    % Connect every node in the hidden layer with the output layer
    \foreach \source in {1,...,5}
        \path (H-\source) edge (O);

    % Annotate the layers
    \node[annot,above of=H-1, node distance=1cm] (hl) {Capa Oculta};
    \node[annot,left of=hl] {Capa de entrada};
    \node[annot,right of=hl] {Capa de salida};
\end{tikzpicture}
% End of code
\end{document}
