\documentclass[12pt,spanish,fleqn,openany,letterpaper,pagesize]{scrbook}

\usepackage[utf8]{inputenc}
\usepackage[spanish]{babel}
\usepackage{fancyhdr}
\usepackage{epsfig}
\usepackage{epic}
\usepackage{eepic}
\usepackage{amsmath}
\usepackage{amssymb}
\usepackage{siunitx}
\usepackage{booktabs}
\usepackage{lineno}
\usepackage{threeparttable}
\usepackage{amscd}
\usepackage{here}
\usepackage{graphicx}
\usepackage{lscape}
\usepackage{tabularx}
\usepackage{url}
\usepackage{subfigure}
\usepackage{longtable}
\usepackage{framed}
\usepackage{ragged2e} \justifying
\usepackage{rotating} %Para rotar texto, objetos y tablas seite. No se ve en DVI solo en PS. Seite 328 Hundebuch
                        %se usa junto con \rotate, \sidewidestable ....
\usepackage{tikz}
\usepackage{verbatim}


\usepackage{bbm}

\usepackage{listings}
\usepackage{color}

\definecolor{dkgreen}{rgb}{0,0.6,0}
\definecolor{gray}{rgb}{0.5,0.5,0.5}
\definecolor{mauve}{rgb}{0.58,0,0.82}

\lstset{frame=tb,
  language=Bash,
  aboveskip=3mm,
  belowskip=3mm,
  showstringspaces=false,
  columns=flexible,
  basicstyle={\small\ttfamily},
  numbers=none,
  numberstyle=\tiny\color{gray},
  keywordstyle=\color{blue},
  commentstyle=\color{dkgreen},
  stringstyle=\color{mauve},
  breaklines=true,
  breakatwhitespace=true,
  tabsize=3
}

\renewcommand{\theequation}{\thechapter-\arabic{equation}}
\renewcommand{\thefigure}{\textbf{\thechapter-\arabic{figure}}}
\renewcommand{\thetable}{\textbf{\thechapter-\arabic{table}}}
\newcommand{\norm}[1]{\left\lVert#1\right\rVert}

\pagestyle{fancyplain}%\addtolength{\headwidth}{\marginparwidth}
\textheight22.5cm \topmargin0cm \textwidth16.5cm
\oddsidemargin0.5cm \evensidemargin-0.5cm%
\renewcommand{\chaptermark}[1]{\markboth{\thechapter\; #1}{}}
\renewcommand{\sectionmark}[1]{\markright{\thesection\; #1}}
\lhead[\fancyplain{}{\thepage}]{\fancyplain{}{\rightmark}}
\rhead[\fancyplain{}{\leftmark}]{\fancyplain{}{\thepage}}
\fancyfoot{}
\thispagestyle{fancy}%


\addtolength{\headwidth}{0cm}
\unitlength1mm %Define la unidad LE para Figuras
\mathindent0cm %Define la distancia de las formulas al texto,  fleqn las descentra
\marginparwidth0cm
\parindent0cm %Define la distancia de la primera linea de un parrafo a la margen

%Para tablas,  redefine el backschlash en tablas donde se define la posici\'{o}n del texto en las
%casillas (con \centering \raggedright o \raggedleft)
\newcommand{\PreserveBackslash}[1]{\let\temp=\\#1\let\\=\temp}
\let\PBS=\PreserveBackslash

%Espacio entre lineas
\renewcommand{\baselinestretch}{1.1}

%Neuer Befehl f\"{u}r die Tabelle Eigenschaften der Aktivkohlen
\newcommand{\arr}[1]{\raisebox{1.5ex}[0cm][0cm]{#1}}

%Neue Kommandos
\usepackage{Befehle}


%Trennungsliste
\hyphenation {Reaktor-ab-me-ssun-gen Gas-zu-sa-mmen-set-zung
Raum-gesch-win-dig-keit Durch-fluss Stick-stoff-gemisch
Ad-sorp-tions-tem-pe-ra-tur Klein-schmidt
Kohlen-stoff-Mole-kular-siebe Py-rolysat-aus-beu-te
Trans-port-vor-gan-ge}


\begin{document}

\section{Epidemiología Panorámica}

\justifying

\subsection{Teledetección}

\par La \textit{Teledetección} se define como el proceso de adquirir
  información acerca de un objeto, área o fenómeno desde la distancia.
  Un sensor remoto es un instrumento capaz de realizar percepción remota, por lo
  que en esta amplia definición caben desde los ojos hasta los
  radiotelescopios.

\par Existen dos grandes tipos de sensores remotos (SR): activos y pasivos.
  Los activos son aquellos que obtienen la información generando su propia energía
  mientras que los pasivos dependen de una fuente externa, que en la Tierra
  principalmente proviene del Sol. Hasta el día de hoy, los más usados son los
  sensores pasivos dado que permiten medir la magnitud de la radiación electromagnética
  reflejada e irradiada desde la superficie de la Tierra y de la atmósfera y,
  de esta manera, derivar información sobre las condiciones de la superficie \cite{cami_tartagal}.


\par Los SR más utilizados y con mayor cantidad de aplicaciones son los que se
  encuentran a bordo de satélites que orbitan sobre la Tierra, bien sea
  en órbitas geoestacionarias\footnote{Están en altitudes entre 23000 y 40000 km,
  sobre la franja ecuatorial y viaja a la misma velocidad de rotación de la Tierra
  por lo que siempre están fijos sobre un punto determinado de la superficie terrestre},
  o aquellos que pasan repetidamente por diferentes secciones de la Tierra mientras
  rotan dado que están orbitando alrededor del planeta a altitudes menores.


\par Las tecnologías relacionadas al ámbito aeroespacial (CHAART) dieron lugar a programas
  que integran estas tecnologías con, por ejemplo, la agricultura, salud pública,
  geología y las ciencias forestales.
  A su vez, la información obtenida por dichos SR se puede aplicar a estudios
  entomológicos\footnote{De insectos}, debido a que ellos proveen gran cantidad
  y diversidad de información sobre la cobertura de la Tierra: características
  de la vegetación, cuerpos de agua, temperaturas, entre otras. Ésta, también es
  información sobre el hábitat de insectos y artrópodos vectores \cite{estallo_ndwi, data_driven_prediction},
  y, por lo tanto, de acuerdo a la teoría de Pavlovsky \cite{pavloskiy} en la que
  expone la correlación entre el hábitat y enfermedades transmitidas por vectores,
  los datos de los SR se pueden utilizar como fuente de información sobre la
  distribución espacio-temporal de dichas afecciones.


\par Con la acumulación de datos registrados por sensores remotos desde los años
  70 existen series temporales que permiten realizar varios tipos de análisis con
  relevancia para la transmisión de la enfermedad de Chagas y otras
  ETV\footnote{Enfermedades de Transmisión Vectorial}.
  Entre ellas, series temporales de imágenes de mediana resolución espacial
  permiten analizar en perspectiva histórica los cambios de uso y cobertura del
  terreno, proceso que habitualmente tiene vinculación con cambios en la
  epidemiología de la enfermedad \cite{german_temporal}.
  A su vez, el deterioro de las condiciones de salud en el mundo, el avance significativo
  en el procesamiento de computadoras, la mejora en la adquisición de datos,
  la reducción de los costos de hardware y software y el desarrollo de tecnología
  GIS\footnote{Sistema de Información Geográfica} han llevado al lanzamiento
  de programas que apuntan a integrar SR / GIS en aplicaciones de salud por CHAART
  \cite{tesis_riesgo_viral, tesis_gonza, espinosa_temporal, rs_public_health}.


\subsection{Epidemiología Panorámica}


\par El uso de técnicas de Teledetección para mapear la distribución de vectores y el riesgo
  de enfermedades ha tenido una gran evolución durante las últimas dos
  décadas (13). La complejidad de las técnicas va desde el uso de simples
  correlaciones entre las firmas espectrales de diferentes coberturas, usos del
  suelo y abundancia de especies (26,27) hasta técnicas complejas que integran variables
  ambientales obtenidas de satélites con la biología de los vectores (13).
  Estas técnicas se usan para desarrollar modelos predictivos de riesgo,
  los cuales principalmente se realizan a través de técnicas estadísticas de
  regresión logística y análisis discriminante, que dilucidan las asociaciones
  entre datos ambientales multivariados y los patrones de presencia o ausencia de
  vectores para así mapear los vectores o las enfermedades.
  Estos métodos son capaces de predecir la probabilidad “a posteriori” de la
  presencia de la variable dependiente (vector o enfermedad), a partir de un
  grupo de variables independientes (datos de clima y cobertura de la tierra) y de esta
  forma pueden ser usados para hacer mapas de riesgo a partir de bases de datos.


\par Por definición, la \textbf{\textit{Epidemiología Panorámica}} integra conceptos y
  enfoques de la ecología vinculada a las enfermedades, con el análisis a
  macroescala de la ecología del paisaje. La intersección de estas perspectivas
  nos habilita a entender cómo es que la configuración espacial y las
  características de la composición del paisaje afectan a los procesos
  epidemiológicos a lo largo y ancho de las áreas geográficas que se
  extienden más allá de los procesos que operan localmente dentro de una sola comunidad.
  Así, la \textit{Epidemiología Panorámica} es más que simplemente establecer
  sectores en el territorio y examinar diferencias en condiciones locales de
  factores bióticos y abióticos entre distintos lugares; la clave es obtener
  información sobre la distribución geográfica de la enfermedad y comprender
  cómo las interrelaciones de los paisajes influencian las intereacciones entre
  individuos susceptibles e infectados.

Los SR han sido aplicados en gran variedad de estudios sobre vectores de
enfermedades (28-37). Por ejemplo, en México, Dumonteil y Gourbiere (38)
estudiaron la relación entre la distribución de la especie Triatoma dimidiata
y factores bioclimáticos, para de esta forma desarrollar un modelo predictivo de
la abundancia domiciliaria por esta especie y las tasas de infección
por T. cruzi. Estas predicciones se usaron para construir el primer mapa de
riesgo de transmisión en la península de Yucatán hallándose que la abundancia de T.
dimidiata se asocia de forma positiva (por análisis de regresión de Poisson)
con los cultivos, pastos, precipitación, humedad relativa y la temperatura
máxima. [poner ejemplo de trabajos en Argentina]

-------------------------------------------

Epidemiología Panorámica [21, 22] promueve la noción de que la información
satelital derivada de la observación de la Tierra y las tecnologías geoespaciales
son herramientas escenciales [23] para abordar los problemas epidemiológicos
relacionados a la transmisión de enfermedades por vectores. Utilizando
estas ideas, muchos estudios interdisciplinarios fueron llevados a cabo en
latinoamérica enfocados en la generación de modelos predictivos de riesgo,
espaciales y temporales, basados en condiciones ambientales derivadas de
información satelital [24, 25, 26, 27]. En particular, en Argentina
existen varias experiencias en esta dirección, [28, 29, 30] abordan el
problema de la epidemia del Dengue dando herramientas operacionales [31].
A nivel global, también se pueden encontrar contribuciones en esta área
[32, 33, 34] también con algunas experincias de herramientas operativas [35]


----------------------------------------

El concepto interdisciplinario de \textit{Epidemiología Panorámica} (EP)
ha permitido generar algunos progresos en la epidemiología del dengue y otras
enfermedades transmitidas por vectores y zoonóticas como el chagas, la
malaria, la leishmaniasis y el hantavirus. Esta herramienta, EP, se centra en
producir mapas de riesgo predictivos espaciales y temporales basados en
características ambientales junto con datos de campo [7-14].
Estos estudios en Latinoamérica, donde se utiliza la tecnología espacial en
problemas epidemiológicos, están inspirados en trabajos pioneros llevados a cabo
en EE. UU. y Europa [15-18].
Con base en los trabajos citados en el párrafo anterior, en 2011, Argentina
comenzó a desarrollar un proyecto operacional
(Sistema de Alerta Temprana de Salud, HEWS), útil tanto para las autoridades de
salud como para los investigadores.


Básicamente, HEWS es un mapeo de riesgo dinámico del dengue para todas las
ciudades del país. En este producto, cada ciudad es representada por un punto
al que se le asigna un valor de riesgo para cada año, basado en tecnología
geoespacial. El trabajo fue realizado en un contexto interdisciplinario e
interinstitucional.

En este sistema [13], el riesgo se evalúa en cuatro componentes que son: el
entomológico, el viral, el componente relacionado con las actividades de
control y finalmente el ambiental. Mientras que los tres primeros componentes
se generan con el aporte de información de los agentes de salud que trabajan en cada
ciudad, el cuarto se evalúa a partir de datos satelitales.

Específicamente el componente ambiental, en la versión inicial del sistema, se
evalúa con una probabilidad estacionaria de presencia de vectores (igual para
todo el tiempo) más un componente relacionado con el número de ciclos virales,
que son una función de la temperatura, y así es diferente para cada ciudad y
para cada año. El mapa de probabilidad de presencia de especie (modelo de nicho)
es claramente una gran simplificación y se puede mejorar en base a datos
satelitales continuos del medio ambiente. Variables como precipitación y
temperatura, han demostrado, con una variabilidad local, influenciar el
desarrollo de mosquitos, su supervivencia y actividad de oviposición y por ende
la abundancia de vectores.

En particular trabajos previos como los de Estallo [8], [19] utilizan información
satelital para estimar la evolución temporal de la abundancia de vectores.
En 2017 German y colaboradores [20] desarrollan una metodología completa
para generar modelos de manera automática y basada en información de libre
acceso. En particular German [20] utiliza productos del sensor (MODIS) a bordo
del satélite Terra y Aqua, pues es uno de los más adecuados para esta
aplicación particular, debido a su resolución temporal, espectral y espacial.
MODIS proporciona un conjunto de productos pre-procesados y de libre acceso [21].
Específicamente, los productos de vegetación (índice de vegetación de diferencia
normalizada) y temperatura (temperatura de la superficie terrestre) derivados
de MODIS son ejemplos de variables de percepción remota utilizadas
en aplicaciones de epidemiologia [13], [22] incorporadas en [20]. Otra variable
ambiental obtenida de satélite que es relevante e incorporada, es el
\textit{Índice de Agua de Diferencia Normalizada} (NDWI) que evalúa de alguna
forma el contenido de agua de la cubierta. Adicionalmente el trabajo de German
incorpora una estimación de la precipitación desde el espacio a partir de las
misiones (TRMM) y (GPM) [23].



--------------------------------------------------------------------------


La Epidemiología Panorámica está estrechamente relacionada a su paralela
ecológica, la Ecología Panorámica, una ciencia con inicios en los años
1930s que se dedica a estudiar las interacciones entre los ambientes y la
vegetación (122).
Sin embargo, los paisajes son espacial y temporalmente dinámicos.
En simultáneo con el nacimiento de la ecología Panorámica como una ciencia,
Pavloskiy estipula el concepto de \textit{nidalidad}[agregar footnote] (o focalidad) de las
enfermedades (86, 87), donde los patógenos son asociados a paisajes (zonas)
especificos. Un foco de infección contiene tres elementos críticos
\begin{enumerate}
  \item Vectores con capacidad de transmisión de la infección
  \item Vertebrados capaces de funcionar como reservorio de la infección
  \item Anfitriones susceptibles, como humanos o animales domésticos
\end{enumerate}

El concepto de \textit{focalidad} mezclado con la ecología panorámica
llevó al nacimiento de la ciencia contemporánea
\textbf{\textit{Epidemiología Panorámica}} (37), en la cual las enfermedades
pueden ser asociadas a distintas características del paisaje o cómo
la configuración entre el vector, el huesped y el patógeno se intersecan
dada un clima permisivo para que ello suceda.

-----------------------------------------
\begin{figure}
\centering%
\includegraphics[width=1\textwidth]{images/paisajes_heterogeneos}%
\caption{La propagación y persistencia a través de paisajes heterogéneos}\label{fig:paisajes_h}
\end{figure}


Las condiciones ambientales que determinan la conectividad de los paisajes
para la disperción pueden variar en las distintas regiones y dependen de cómo
el patógeno se dispersa biológicamente (e.j: dado un patógeno portado de
vectores, el movimiento del insecto) o abióticamente (e.j: flujos de viento o agua).
Por ejemplo, rios y corrientes pueden actuar como corredores de disperción que
fomentan la propagación de la infección a través de paisajes heterogéneos
para patógenos de plantas transmitidos por el agua (53), mientras que en otros
sistemas, como las enfermedades zoonóticas de mamíferos terrestres, estos mismos
cuerpos de agua pueden funcionar como barreras impidiendo el movimiento
del huesped o del vector (97). Éstas condiciones se ven reflejadas en la
Figura \ref{fig:paisajes_h} de \cite{Landscape Epidemiology of Emerging Infectious...}. Notemos que en el caso de \textbf{a)}, la
conectividad entre los sitios azules es mayor que la que se da entre éstos y
los amarillos, y también entre ellos y los rojos, siendo que la distancia
Euclidea entre los azules. Ésto ocurre porque el sitio rojo está del otro lado
de la cordillera, la cual funciona como una barrera geográfica para la
inoculación\footnote{Introducción de microorganismos vivos, muertos o atenuados,
en un organismo de forma accidental o voluntaria.}, el huésped y/o la disperción
del vector. En el caso \textbf{b)}, en cambio, se da la situación de un
patosistema\footnote{Subsistema dentro del sistema agrícola caracterizado por el
fenómeno de parasitismo. Está constituido por un hospedante susceptible,
un patógeno virulento y un ambiente predispuesto a la enfermedad} acuático, en
donde la inoculación sucede a través del agua: los dos sitios amarillos son los
más estrechamente conectados, a pesar de que están separados por una mayor
distancia Euclidea que con otros, porque el sitio amarillo de abajo está localizado
bajo una corriente que va desde el sitio amarillo de arriba.



Implementar el enfoque panorámico es una tarea no trivial. Los enfoques
panorámicos que incorporan complejidad espacio-temporal a los sistemas
epidemiológicos requieren un cuidado al establecer relaciones entre las
observaciones moleculares y microbianas de la distribución de las enfermedades
a través de mediciones de las condiciones bióticas y abióticas
correspondientes (y circundantes) (7).


\end{document}
