\documentclass[12pt,spanish,fleqn,openany,letterpaper,pagesize]{scrbook}

\usepackage[utf8]{inputenc}
\usepackage[spanish]{babel}
\usepackage{fancyhdr}
\usepackage{epsfig}
\usepackage{epic}
\usepackage{eepic}
\usepackage{amsmath}
\usepackage{threeparttable}
\usepackage{amscd}
\usepackage{here}
\usepackage{graphicx}
\usepackage{lscape}
\usepackage{tabularx}
\usepackage{subfigure}
\usepackage{longtable}


\usepackage{rotating} %Para rotar texto, objetos y tablas seite. No se ve en DVI solo en PS. Seite 328 Hundebuch
                        %se usa junto con \rotate, \sidewidestable ....


\renewcommand{\theequation}{\thechapter-\arabic{equation}}
\renewcommand{\thefigure}{\textbf{\thechapter-\arabic{figure}}}
\renewcommand{\thetable}{\textbf{\thechapter-\arabic{table}}}


\pagestyle{fancyplain}%\addtolength{\headwidth}{\marginparwidth}
\textheight22.5cm \topmargin0cm \textwidth16.5cm
\oddsidemargin0.5cm \evensidemargin-0.5cm%
\renewcommand{\chaptermark}[1]{\markboth{\thechapter\; #1}{}}
\renewcommand{\sectionmark}[1]{\markright{\thesection\; #1}}
\lhead[\fancyplain{}{\thepage}]{\fancyplain{}{\rightmark}}
\rhead[\fancyplain{}{\leftmark}]{\fancyplain{}{\thepage}}
\fancyfoot{}
\thispagestyle{fancy}%


\addtolength{\headwidth}{0cm}
\unitlength1mm %Define la unidad LE para Figuras
\mathindent0cm %Define la distancia de las formulas al texto,  fleqn las descentra
\marginparwidth0cm
\parindent0cm %Define la distancia de la primera linea de un parrafo a la margen

%Para tablas,  redefine el backschlash en tablas donde se define la posici\'{o}n del texto en las
%casillas (con \centering \raggedright o \raggedleft)
\newcommand{\PreserveBackslash}[1]{\let\temp=\\#1\let\\=\temp}
\let\PBS=\PreserveBackslash

%Espacio entre lineas
\renewcommand{\baselinestretch}{1.1}

%Neuer Befehl f\"{u}r die Tabelle Eigenschaften der Aktivkohlen
\newcommand{\arr}[1]{\raisebox{1.5ex}[0cm][0cm]{#1}}

%Neue Kommandos
\usepackage{Befehle}


%Trennungsliste
\hyphenation {Reaktor-ab-me-ssun-gen Gas-zu-sa-mmen-set-zung
Raum-gesch-win-dig-keit Durch-fluss Stick-stoff-gemisch
Ad-sorp-tions-tem-pe-ra-tur Klein-schmidt
Kohlen-stoff-Mole-kular-siebe Py-rolysat-aus-beu-te
Trans-port-vor-gan-ge}


\begin{document}

\section{Epidemiología Panorámica}

\justifying

Epidemiología Panorámica [21, 22] promueve la noción de que la información
satelital derivada de la observación de la Tierra y las tecnologías geoespaciales
son herramientas escenciales [23] para abordar los problemas epidemiológicos
relacionados a la transmisión de enfermedades por vectores. Utilizando
estas ideas, muchos estudios interdisciplinarios fueron llevados a cabo en
latinoamérica enfocados en la generación de modelos predictivos de riesgo,
espaciales y temporales, basados en condiciones ambientales derivadas de
información satelital [24, 25, 26, 27]. En particular, en Argentina
existen varias experiencias en esta dirección, [28, 29, 30] abordan el
problema de la epidemia del Dengue dando herramientas operacionales [31].
A nivel global, también se pueden encontrar contribuciones en esta área
[32, 33, 34] también con algunas experincias de herramientas operativas [35]


----------------------------------------

El concepto interdisciplinario de \textit{Epidemiología Panorámica} (EP)
ha permitido generar algunos progresos en la epidemiología del dengue y otras
enfermedades transmitidas por vectores y zoonóticas como el chagas, la
malaria, la leishmaniasis y el hantavirus. Esta herramienta, EP, se centra en
producir mapas de riesgo predictivos espaciales y temporales basados en
características ambientales junto con datos de campo [7-14].
Estos estudios en Latinoamérica, donde se utiliza la tecnología espacial en
problemas epidemiológicos, están inspirados en trabajos pioneros llevados a cabo
en EE. UU. y Europa [15-18].
Con base en los trabajos citados en el párrafo anterior, en 2011, Argentina
comenzó a desarrollar un proyecto operacional
(Sistema de Alerta Temprana de Salud, HEWS), útil tanto para las autoridades de
salud como para los investigadores.


Básicamente, HEWS es un mapeo de riesgo dinámico del dengue para todas las
ciudades del país. En este producto, cada ciudad es representada por un punto
al que se le asigna un valor de riesgo para cada año, basado en tecnología
geoespacial. El trabajo fue realizado en un contexto interdisciplinario e
interinstitucional.

En este sistema [13], el riesgo se evalúa en cuatro componentes que son: el
entomológico, el viral, el componente relacionado con las actividades de
control y finalmente el ambiental. Mientras que los tres primeros componentes
se generan con el aporte de información de los agentes de salud que trabajan en cada
ciudad, el cuarto se evalúa a partir de datos satelitales.

Específicamente el componente ambiental, en la versión inicial del sistema, se
evalúa con una probabilidad estacionaria de presencia de vectores (igual para
todo el tiempo) más un componente relacionado con el número de ciclos virales,
que son una función de la temperatura, y así es diferente para cada ciudad y
para cada año. El mapa de probabilidad de presencia de especie (modelo de nicho)
es claramente una gran simplificación y se puede mejorar en base a datos
satelitales continuos del medio ambiente. Variables como precipitación y
temperatura, han demostrado, con una variabilidad local, influenciar el
desarrollo de mosquitos, su supervivencia y actividad de oviposición y por ende
la abundancia de vectores.

En particular trabajos previos como los de Estallo [8], [19] utilizan información
satelital para estimar la evolución temporal de la abundancia de vectores.
En 2017 German y colaboradores [20] desarrollan una metodología completa
para generar modelos de manera automática y basada en información de libre
acceso. En particular German [20] utiliza productos del sensor (MODIS) a bordo
del satélite Terra y Aqua, pues es uno de los más adecuados para esta
aplicación particular, debido a su resolución temporal, espectral y espacial.
MODIS proporciona un conjunto de productos pre-procesados y de libre acceso [21].
Específicamente, los productos de vegetación (índice de vegetación de diferencia
normalizada) y temperatura (temperatura de la superficie terrestre) derivados
de MODIS son ejemplos de variables de percepción remota utilizadas
en aplicaciones de epidemiologia [13], [22] incorporadas en [20]. Otra variable
ambiental obtenida de satélite que es relevante e incorporada, es el
\textit{Índice de Agua de Diferencia Normalizada} (NDWI) que evalúa de alguna
forma el contenido de agua de la cubierta. Adicionalmente el trabajo de German
incorpora una estimación de la precipitación desde el espacio a partir de las
misiones (TRMM) y (GPM) [23].

---------------------------------------------

La \textbf{\textit{Teledetección}} (RS, sus siglas en inglés) permite a los científicos el
estudio de componentes bióticos y abióticos de la superficie terrestre.
La RS, esencialmente, es la medición the energía reflejada o emitida por los
cuerpos en distintos espectros electromagnéticos utilizando sensores.
En el caso de la Teledetección espacial, dichos sensores se presentan a bordo
de satelites.
Las tecnologías relacionadas al ámbito aeroespacial (CHAART) dieron lugar a programas
que integran estas tecnologías con, por ejemplo, la agricultura, salud pública,
geología y las ciencias forestales.

El deterioro de las condiciones de salud en el mundo, el avance significativo
en el procesamiento de computadoras, la mejora en la
adquisición de datos, la reducción de los costos de hardware y software y el
desarrollo de tecnología GIS basada en computadora han llevado al lanzamiento
de programas que apuntan a integrar RS / GIS en aplicaciones de salud por CHAART.


--------------------------------------------



Sensores remotos

La percepción remota se define como el proceso de adquirir información acerca
de un objeto, área o fenómeno desde la distancia. Esta amplia definición cubre
prácticamente todo, desde los ojos hasta los radiotelescopios. Los sensores
remotos (SR) se pueden categorizar como activos o pasivos, diferenciándose por la fuente
de energía de la cual se obtiene la información.
Los sensores activos generan su propia energía, mientras que los pasivos
dependen de energía ambiental de una fuente externa, que en la tierra
proviene principalmente del sol. Los más usados son los sensores pasivos,
que permiten medir la magnitud de la radiación electromagnética reflejada e
irradiada desde la superficie de la tierra y de la atmósfera y, así mismo,
derivar información sobre las condiciones de la superficie (3).

Los SR de más amplio uso y con mayores aplicaciones son aquellos instalados a
bordo de satélites que orbitan sobre la tierra, bien sea en orbitas geoestacionarias (en
altitudes de 23 000 y 40 000 km) sobre la franja ecuatorial y que viajan a la
misma velocidad de rotación de la tierra, lo que permite que siempre estén
fijos sobre un punto determinado de la superficie terrestre, o aquellos que
están orbitando alrededor del planeta a altitudes menores (600-900 km) los
cuales pasan repetidas veces por diferentes secciones de la tierra mientras
rotan, a estos satélites se les denomina de tipo polar (3).


La información obtenida por los SR se puede aplicar a estudios entomológicos de
campo, debido a que ellos proveen información importante sobre la cobertura de
la tierra: tipos de vegetación, cuerpos de agua, temperatura de la superficie,
temperatura del aire, etc. o sea, información acerca del hábitat de los insectos
o artrópodos vectores (4); por lo tanto, y de acuerdo a la teoría de
Pavlovsky (17) la cual expone la correlación entre hábitat y enfermedades
transmitidas por vectores, los datos obtenidos de SR se pueden usar como
fuente de información sobre la distribución espacial de los vectores y de las
enfermedades.


Existe un número de variables ambientales que tienen influencia directa o
indirecta sobre la dinámica poblacional de los vectores. Muchas de ellas
pueden estimarse a partir de los datos registrados por sensores a bordo de plataformas
en órbita espacial. Entre tales variables pueden mencionarse: temperatura del
aire, temperatura de superficie, índice de vegetación de diferencia normalizada
(NDVI, por sus siglas en inglés), radiación infrarroja media, déficit de saturación
de vapor.



Con la acumulación de datos registrados por sensores remotos desde los años
70 existen series temporales que permiten realizar varios tipos de análisis con
relevancia para la transmisión de la enfermedad de Chagas y otras ETV.
Entre ellas, series temporales de imágenes de mediana resolución espacial
permiten analizar en perspectiva histórica los cambios de uso y cobertura del
terreno, proceso que habitualmente tiene vinculación con cambios en la
epidemiología de la enfermedad (25).

El uso de técnicas de SR para mapear la distribución de vectores y el riesgo
de enfermedades ha tenido una gran evolución durante las últimas dos
décadas (13). La complejidad de las técnicas va desde el uso de simples
correlaciones entre las firmas espectrales de diferentes coberturas, usos del
suelo y abundancia de especies (26,27) hasta técnicas complejas que integran variables
ambientales obtenidas de satélites con la biología de los vectores (13).
Estas técnicas se usan para desarrollar modelos predictivos de riesgo,
los cuales principalmente se realizan a través de técnicas estadísticas de
regresión logística y análisis discriminante, que dilucidan las asociaciones
entre datos ambientales multivariados y los patrones de presencia o ausencia de
vectores para así mapear los vectores o las enfermedades.
Estos métodos son capaces de predecir la probabilidad “a posteriori” de la
presencia de la variable dependiente (vector o enfermedad), a partir de un
grupo de variables independientes (datos de clima y cobertura de la tierra) y de esta
forma pueden ser usados para hacer mapas de riesgo a partir de bases de datos.

Los SR han sido aplicados en gran variedad de estudios sobre vectores de
enfermedades (28-37). Por ejemplo, en México, Dumonteil y Gourbiere (38)
estudiaron la relación entre la distribución de la especie Triatoma dimidiata
y factores bioclimáticos, para de esta forma desarrollar un modelo predictivo de
la abundancia domiciliaria por esta especie y las tasas de infección
por T. cruzi. Estas predicciones se usaron para construir el primer mapa de
riesgo de transmisión en la península de Yucatán hallándose que la abundancia de T.
dimidiata se asocia de forma positiva (por análisis de regresión de Poisson)
con los cultivos, pastos, precipitación, humedad relativa y la temperatura
máxima. [poner ejemplo de trabajos en Argentina]

--------------------------------------------------------------------------



Landscape epidemiology is linked closely
to its ecological parallel, landscape ecology, a
science with beginnings in the early 1930s to
study interactions between the environment
and vegetation (122). Geomorphology and
climate combine to delineate plant communi-
ties and therefore ecosystem structure. These
biomes were first classified at a global scale in
the early 1900s by Köppen, who proposed that
climatic zones could be effectively defined by
their resulting plant communities. This clas-
sification system was updated recently for the
1951–2000 period (88) and broadly delineates
ecosystems with varying capacity to support
different vectors, hosts, and pathogens. Some
pathogens such as WNV have invaded almost
all but the most extreme climates, whereas
others, such as Barmah Forest virus, seem
more restrictive in their distribution. How-
ever, landscapes are spatially and temporally
dynamic. The increasing size and expanding
distribution of the global human population


have destabilized climax communities that have
been intruded upon by agriculture and human
habitation and must respond to anthropogenic-
driven climate and ecosystem change (31). The
resulting heterogeneous matrix (http://www.
eoearth.org/article/Anthropogenic biomes)
consists of habitat patches that change in di-
mension owing to human and other activities
and are linked by a series of transitional
boundaries (or ecotones) characterized by their
distinct and diverse vegetative types, enhanced
primary productivity, species diversity, and
complex interactions (also referred to as the
edge effect).
Concurrent with the emergence of land-
scape ecology as a science, Pavloskiy proposed
the concept of nidality (or focality) of dis-
ease (86, 87), where pathogens are associated
with specific landscapes. The nidus of infection
contains three critical elements: (a) competent
and infectious vectors, (b) competent vertebrate
reservoirs, and (c) susceptible recipient (tangen-
tial) hosts such as humans or domestic animals.
In his original concept based on tick-borne
pathogens in Russia, humans became infected
when they traveled into the nidus and contacted
the infectious vector or perhaps the reservoir
host. The disjunct distribution of infection and
residence complicated the initial understanding
of the epidemiology of transmission, especially
for highly clustered tick-borne pathogens such
as tick-borne encephalitis virus (83). This nidal-
ity concept blended with landscape ecology has
led to the emergence of the contemporary sci-
ence of landscape epidemiology (37), in which
diseases may be associated with distinct land-
scape features or ecological settings where vec-
tor, host, and pathogen intersect within a per-
missive climate (Figure 1). When placed into
the recent context of anthropogenically altered
biomes or anthromes (31), the nidus now ranges
from remote, sparsely inhabited landscapes to
residential settings and may be delineated by
various mapping and statistical tools (56). For
many vector-borne anthroponoses, the nidus
may be the human residence where commensal
vectors reside and blood feed.


==============================================

Landscape Epidemiology [21, 22] promotes the notion that satellite data
from earth observation and geospatial technologies are essential tools [23] to
address vector borne epidemiological problems. Using these ideas, several
interdisciplinary studies were produced in latinoamerica focused in generat-
ing spatial and temporal predictive risk models based on satellite derived
environmental conditions [24, 25, 26, 27]. In particular in Argentine there
are some interesting experiences on this issue, [28, 29, 30] deal with Dengue
epidemics leading to operational tools [31]. At a global scope we can find
interesting contributions [32, 33, 34] with also some operatives experiences
[35]


El concepto interdisciplinario de \textit{Epidemiología Panorámica} (EP)
ha permitido generar algunos progresos en la epidemiología del dengue y otras
enfermedades transmitidas por vectores y zoonóticas como el chagas, la
malaria, la leishmaniasis y el hantavirus. Esta herramienta, EP, se centra en
producir mapas de riesgo predictivos espaciales y temporales basados en
características ambientales junto con datos de campo [7-14].
Estos estudios en Latinoamérica, donde se utiliza la tecnología espacial en
problemas epidemiológicos, están inspirados en trabajos pioneros llevados a cabo
en EE. UU. y Europa [15-18].
Con base en los trabajos citados en el párrafo anterior, en 2011, Argentina
comenzó a desarrollar un proyecto operacional
(Sistema de Alerta Temprana de Salud, HEWS), útil tanto para las autoridades de
salud como para los investigadores.

Básicamente, HEWS es un mapeo de riesgo
dinámico del dengue para todas las ciudades del país. En este producto, cada
ciudad es representada por un punto al que se le asigna un valor de riesgo para
cada año, basado en tecnología geoespacial. El trabajo fue realizado en un
contexto interdisciplinario e interinstitucional.

En este sistema [13], el riesgo se evalúa en cuatro componentes que son: el
entomológico, el viral, el componente relacionado con las actividades de
control y finalmente el ambiental. Mientras que los tres primeros componentes
se generan con el aporte de información de los agentes de salud que trabajan en cada
ciudad, el cuarto se evalúa a partir de datos satelitales.

Específicamente el componente ambiental, en la versión inicial del sistema, se
evalúa con una probabilidad estacionaria de presencia de vectores (igual para
todo el tiempo) más un componente relacionado con el número de ciclos virales,
que son una función de la temperatura, y así es diferente para cada ciudad y
para cada año. El mapa de probabilidad de presencia de especie (modelo de nicho)
es claramente una gran simplificación y se puede mejorar en base a datos
satelitales continuos del medio ambiente. Variables como precipitación y
temperatura, han demostrado, con una variabilidad local, influenciar el
desarrollo de mosquitos, su supervivencia y actividad de oviposición y por ende
la abundancia de vectores.

En particular trabajos previos como los de Estallo [8], [19] utilizan información
satelital para estimar la evolución temporal de la abundancia de vectores.
En 2017 German y colaboradores [20] desarrollan una metodología completa
para generar modelos de manera automática y basada en información de libre
acceso. En particular German [20] utiliza productos del sensor (MODIS) a bordo
del satélite Terra y Aqua, pues es uno de los más adecuados para esta
aplicación particular, debido a su resolución temporal, espectral y espacial.
MODIS proporciona un conjunto de productos pre-procesados y de libre acceso [21].
Específicamente, los productos de vegetación (índice de vegetación de diferencia
normalizada) y temperatura (temperatura de la superficie terrestre) derivados
de MODIS son ejemplos de variables de percepción remota utilizadas
en aplicaciones de epidemiologia [13], [22] incorporadas en [20]. Otra variable
ambiental obtenida de satélite que es relevante e incorporada, es el
\textit{Índice de Agua de Diferencia Normalizada} (NDWI) que evalúa de alguna
forma el contenido de agua de la cubierta. Adicionalmente el trabajo de German
incorpora una estimación de la precipitación desde el espacio a partir de las
misiones (TRMM) y (GPM) [23].



-----------------
Landscape epidemiology: An emerging perspective in the mapping and
modelling of disease and disease risk factors

Remote sensing ( RS ) enables scientists to study the biotic
and abiotic components of the earth surface. RS essentially
measures energy reflected or emitted in distinct and
specific electromagnetic spectrum, using sensors usually
onboard satellites. T hey are used to monitor and observe the
earth’s landscape. T he epidemiological applications of RS
and GIS have been reviewed extensively

S ince the launch of ERTS - 1 ( E arth R esources T echnology
S atellites ) in 1972 , N ational A eronautics and S pace
A dministration ( NASA ’s ) C enter for H ealth A pplications



Aerospace Related Technologies ( CHAART ) has initiated
programs aimed at integrating these technologies into the
areas of forestry, agriculture, geology and public health.
T he worsening health conditions around the world and
the significant advancement in computer processing,
improvement in data acquisition, reduced hardware and
software cost and the development of computer based GIS
technology have led to the launching of programs that aim at
integrating RS / GIS into health applications by CHAART .

T o better apply these technologies and satellite-sensor
capabilities, NASA - CHAART developed a website to
evaluate sensors for health applications. T hey defined
16 groups of physical factors that could be used for both
research and in practice. E ach of these factors is essentially
an environmental variable that might have a direct and
indirect bearing in the survival of pathogen, vectors,
reservoir or host. T hese factors ( biotic and abiotic ) may
also affect many types of non-vector borne diseases such
as water borne diseases. E xamples of these factors include
soil moisture, wetlands, vegetation and crop type and sea
surface temperature. A ll these factors can be sensed by a
sensor mounted on a satellite imagery used for landscape
epidemiology. S ome of the links between these factors and
various diseases as developed by CHAART are listed in

T able 1 .
T wo approaches used for mapping diseases are the
correlation of host distribution with climatic data or
correlation with landscape. M ost human studies using RS
have focused on data obtained from L andsat’s M ultispectral
S canner ( MSS ) and T hematic M apper ( TM ) , the N ational
O ceanic and A tmospheric A dministration ( NOAA ) ’s
A dvanced V ery H igh R esolution R adiometer ( AVHRR ) , and
F rance’s S ystème P our l’ O bservation de la T erre ( SPOT ) .
I n many of these studies, remotely sensed data were used
to derive three variables: landscape structure, vegetation
cover and water bodies. T able 2 contains examples some
researches conducted using remote sensing.

-------------------------


Sensores remotos
La percepción remota se define como el proce-
so de adquirir información acerca de un objeto,
área o fenómeno desde la distancia. Esta amplia
definición cubre prácticamente todo, desde los
ojos hasta los radiotelescopios. Los sensores
remotos (SR) se pueden categorizar como ac-
tivos o pasivos, diferenciándose por la fuente
de energía de la cual se obtiene la información.
Los sensores activos generan su propia energía,
mientras que los pasivos dependen de energía
ambiental de una fuente externa, que en la tierra
proviene principalmente del sol. Los más usados
son los sensores pasivos, que permiten medir la
magnitud de la radiación electromagnética refle-
jada e irradiada desde la superficie de la tierra y
de la atmósfera y, así mismo, derivar información
sobre las condiciones de la superficie (3).
Existe una copiosa información sobre SR, la cual
ha sido ampliamente documentada en revisio-
nes como la de Cracknell en 1998 (23) y Kalluri
y col. en 2007 (24). Los SR de más amplio uso
y con mayores aplicaciones son aquellos insta-
lados a bordo de satélites que orbitan sobre la
tierra, bien sea en orbitas geoestacionarias (en
altitudes de 23 000 y 40 000 km) sobre la franja
ecuatorial y que viajan a la misma velocidad de
rotación de la tierra, lo que permite que siem-
pre estén fijos sobre un punto determinado de
la superficie terrestre, o aquellos que están orbi-
tando alrededor del planeta a altitudes menores
(600-900 km) los cuales pasan repetidas veces
por diferentes secciones de la tierra mientras
rotan, a estos satélites se les denomina de tipo
polar (3).


Los SR también se clasifican entre aquellos de-
nominados de alta resolución espacial y de baja
resolución espacial. En general, se establece
que los SR de alta resolución espacial son aque-
llos que captan información de la superficie de
la tierra de áreas iguales o menores a 1 x 1km, y,
por lo detallada que puede ser esta información,
aumentan sus costos. Los satélites de la serie
Landsat, Spot y Modis están entre los SR de alta
resolución de mayor uso. De otro lado, los SR
denominados de baja resolución, son aquellos
que brindan información de áreas mayores a 1
x 1 km, entre ellos y de amplio uso podemos
citar a aquellos de la serie NOAA-AVHRR (Natio-
nal Oceanic and Atmospheric Administration-Advanced
Very High Resolution Radiometer).

La información obtenida por los SR se puede
aplicar a estudios entomológicos de campo,
debido a que ellos proveen información impor-
tante sobre la cobertura de la tierra: tipos de
vegetación, cuerpos de agua, temperatura de la
superficie, temperatura del aire, etc. o sea, infor-
mación acerca del hábitat de los insectos o ar-
trópodos vectores (4); por lo tanto, y de acuerdo
a la teoría de Pavlovsky (17) la cual expone la co-
rrelación entre hábitat y enfermedades transmi-
tidas por vectores, los datos obtenidos de SR se
pueden usar como fuente de información sobre
la distribución espacial de los vectores y de las
enfermedades.


Existe un número de variables ambientales que
tienen influencia directa o indirecta sobre la di-
námica poblacional de los vectores. Muchas de
ellas pueden estimarse a partir de los datos re-
gistrados por sensores a bordo de plataformas
en órbita espacial. Entre tales variables pueden
mencionarse: temperatura del aire, temperatura
de superficie, índice de vegetación de diferen-
cia normalizada (NDVI, por sus siglas en inglés),
radiación infrarroja media, déficit de saturación
de vapor. Las variables mencionadas, junto con
la altura de terreno representada en modelos de
elevación digital (DEM, por sus siglas en inglés)
de hasta de 90 metros de resolución espacial,
constituyen un conjunto básico que puede usar-
se para hacer una caracterización ambiental del
área de estudio, dada exclusivamente por las ca-
racterísticas espaciales del área (Ver Figura 1).

Con la acumulación de datos registrados por sen-
sores remotos desde los años 70 existen series
temporales que permiten realizar dos tipos de
análisis con relevancia para la transmisión de la
enfermedad de Chagas y otras ETV. Por un lado,
el análisis de series temporales usando el método
de Fourier, permite extraer información importan-
te sobre la dinámica temporal en ciclos menores
o mayores a un año en los valores de la variable
de interés. El análisis de Fourier permite calcular
no sólo los estadísticos descriptivos básicos, sino
además estadísticos que representan la amplitud
y fase de ciclo 1-, 2-, ó 3- anuales, que definen con
mucho detalle el perfil ambiental de cada punto
del espacio. Por otro lado, series temporales de
imágenes de mediana resolución espacial permi-
ten analizar en perspectiva histórica los cambios
de uso y cobertura del terreno, proceso que ha-
bitualmente tiene vinculación con cambios en la
epidemiología de la enfermedad (25).
El uso de técnicas de SR para mapear la distri-
bución de vectores y el riesgo de enfermedades
ha tenido una gran evolución durante las últimas
dos décadas (13). La complejidad de las técnicas
va desde el uso de simples correlaciones entre
las firmas espectrales de diferentes coberturas,
usos del suelo y abundancia de especies (26,27)
hasta técnicas complejas que integran variables
ambientales obtenidas de satélites con la biolo-
gía de los vectores (13). Estas técnicas se usan
para desarrollar modelos predictivos de riesgo,
los cuales principalmente se realizan a través
de técnicas estadísticas de regresión logística y
análisis discriminante, que dilucidan las asocia-
ciones entre datos ambientales multivariados y
los patrones de presencia o ausencia de vecto-
res para así mapear los vectores o las enferme-
dades. Estos métodos son capaces de predecir
la probabilidad “a posteriori” de la presencia de
la variable dependiente (vector o enfermedad), a
partir de un grupo de variables independientes
(datos de clima y cobertura de la tierra) y de esta
forma pueden ser usados para hacer mapas de
riesgo a partir de bases de datos.

Los SR remotos han sido aplicados en gran
variedad de estudios sobre vectores de en-
fermedades (28-37). Por ejemplo, en México,
Dumonteil y Gourbiere (38) estudiaron la rela-
ción entre la distribución de la especie Triato-
ma dimidiata y factores bioclimáticos, para de
esta forma desarrollar un modelo predictivo de
la abundancia domiciliaria por esta especie y
las tasas de infección por T. cruzi. Estas pre-
dicciones se usaron para construir el primer
mapa de riesgo de transmisión en la península
de Yucatán hallándose que la abundancia de T.
dimidiata se asocia de forma positiva (por aná-
lisis de regresión de Poisson) con los cultivos,
pastos, precipitación, humedad relativa y la
temperatura máxima. Costa y col., en 2002 (39)
demuestran la utilidad de los SR para mapear
la distribución y generar mapas de riesgo para
T. brasiliensis a partir de datos obtenidos de SR:
temperatura del aire, radiación infrarroja me-
dia e índices de vegetación

-----------------------------------------------

Landscape epidemiology is linked closely
to its ecological parallel, landscape ecology, a
science with beginnings in the early 1930s to
study interactions between the environment
and vegetation (122). Geomorphology and
climate combine to delineate plant communi-
ties and therefore ecosystem structure. These
biomes were first classified at a global scale in
the early 1900s by Köppen, who proposed that
climatic zones could be effectively defined by
their resulting plant communities. This clas-
sification system was updated recently for the
1951–2000 period (88) and broadly delineates
ecosystems with varying capacity to support
different vectors, hosts, and pathogens. Some
pathogens such as WNV have invaded almost
all but the most extreme climates, whereas
others, such as Barmah Forest virus, seem
more restrictive in their distribution. How-
ever, landscapes are spatially and temporally
dynamic. The increasing size and expanding
distribution of the global human population


have destabilized climax communities that have
been intruded upon by agriculture and human
habitation and must respond to anthropogenic-
driven climate and ecosystem change (31). The
resulting heterogeneous matrix (http://www.
eoearth.org/article/Anthropogenic biomes)
consists of habitat patches that change in di-
mension owing to human and other activities
and are linked by a series of transitional
boundaries (or ecotones) characterized by their
distinct and diverse vegetative types, enhanced
primary productivity, species diversity, and
complex interactions (also referred to as the
edge effect).
Concurrent with the emergence of land-
scape ecology as a science, Pavloskiy proposed
the concept of nidality (or focality) of dis-
ease (86, 87), where pathogens are associated
with specific landscapes. The nidus of infection
contains three critical elements: (a) competent
and infectious vectors, (b) competent vertebrate
reservoirs, and (c) susceptible recipient (tangen-
tial) hosts such as humans or domestic animals.
In his original concept based on tick-borne
pathogens in Russia, humans became infected
when they traveled into the nidus and contacted
the infectious vector or perhaps the reservoir
host. The disjunct distribution of infection and
residence complicated the initial understanding
of the epidemiology of transmission, especially
for highly clustered tick-borne pathogens such
as tick-borne encephalitis virus (83). This nidal-
ity concept blended with landscape ecology has
led to the emergence of the contemporary sci-
ence of landscape epidemiology (37), in which
diseases may be associated with distinct land-
scape features or ecological settings where vec-
tor, host, and pathogen intersect within a per-
missive climate (Figure 1). When placed into
the recent context of anthropogenically altered
biomes or anthromes (31), the nidus now ranges
from remote, sparsely inhabited landscapes to
residential settings and may be delineated by
various mapping and statistical tools (56). For
many vector-borne anthroponoses, the nidus
may be the human residence where commensal
vectors reside and blood feed.


--------------------------------------

Landscape Epidemiology
of Emerging Infectious
Diseases in Natural and
Human-Altered Ecosystems


By definition,
landscape epidemiology integrates concepts
and approaches from disease ecology with the
macroscale lens of landscape ecology. The
intersection of these perspectives enables us to
understand how the spatial configuration and
composition of landscape features influence
epidemiological processes across broad geo-
graphical areas that extend beyond processes
operating locally within a single community.
Thus, landscape epidemiology is more than
simply establishing plots in the field and ex-
amining differences in local biotic and abiotic
conditions among sites; the key is to gain insight
into the geographical distribution of disease
and to understand how landscape connectivity
influences spatial interactions between sus-
ceptible and infected individuals (Figure 2).
The environmental conditions that determine
landscape connectivity for dispersal may vary
by region and depend on whether a pathogen
disperses biotically (e.g., vector-borne insect
movement) or abiotically (e.g., flows of wind
and water). For example, rivers and streams
may act as dispersal corridors that foster the
spread of infection across a heterogeneous
landscape for waterborne plant pathogens (53),
yet in other systems, such as zoonotic diseases
of terrestrial mammals, these same water
bodies might function as geographic barriers
by impeding host or vector movement (97).
Implementing a landscape approach is typ-
ically not a trivial task. Landscape approaches
that incorporate spatiotemporal complexity in
epidemiological systems require careful spatial
linking of molecular and microbial observa-
tions of disease distribution with measurements
of corresponding (and surrounding) biotic and
abiotic conditions (7). Most landscape epidemi-
ological studies utilize geographic information
systems (GIS) and other geospatial technolo-
gies (e.g., remote sensing) to assimilate the
large, spatial data sets that enable analysis
of relationships between the distribution of
disease and landscape heterogeneity. However,
despite the many ways GIS has helped advance
landscape epidemiology, GIS software is no-
toriously limited to providing static snapshots
of spatial variables in a disease system, with
relatively little ability to incorporate the types
of dynamic temporal complexities (e.g., dis-
persal and infection) needed for process-based
understanding of epidemic behavior over time.



--------------------------------

Modeling Dengue Vector Population Using Remotely
Sensed Data and Machine Learning


Landscape Epidemiology [21, 22] promotes the notion that satellite data
from earth observation and geospatial technologies are essential tools [23] to
address vector borne epidemiological problems. Using these ideas, several
interdisciplinary studies were produced in latinoamerica focused in generat-
ing spatial and temporal predictive risk models based on satellite derived
environmental conditions [24, 25, 26, 27]. In particular in Argentine there
are some interesting experiences on this issue, [28, 29, 30] deal with Dengue
epidemics leading to operational tools [31]. At a global scope we can find
interesting contributions [32, 33, 34] with also some operatives experiences
[35]




\end{document}
