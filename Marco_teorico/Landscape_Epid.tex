\documentclass[12pt,spanish,fleqn,openany,letterpaper,pagesize]{scrbook}

\usepackage[utf8]{inputenc}
\usepackage[spanish]{babel}
\usepackage{fancyhdr}
\usepackage{epsfig}
\usepackage{epic}
\usepackage{eepic}
\usepackage{amsmath}
\usepackage{threeparttable}
\usepackage{amscd}
\usepackage{here}
\usepackage{graphicx}
\usepackage{lscape}
\usepackage{tabularx}
\usepackage{subfigure}
\usepackage{longtable}


\usepackage{rotating} %Para rotar texto, objetos y tablas seite. No se ve en DVI solo en PS. Seite 328 Hundebuch
                        %se usa junto con \rotate, \sidewidestable ....


\renewcommand{\theequation}{\thechapter-\arabic{equation}}
\renewcommand{\thefigure}{\textbf{\thechapter-\arabic{figure}}}
\renewcommand{\thetable}{\textbf{\thechapter-\arabic{table}}}


\pagestyle{fancyplain}%\addtolength{\headwidth}{\marginparwidth}
\textheight22.5cm \topmargin0cm \textwidth16.5cm
\oddsidemargin0.5cm \evensidemargin-0.5cm%
\renewcommand{\chaptermark}[1]{\markboth{\thechapter\; #1}{}}
\renewcommand{\sectionmark}[1]{\markright{\thesection\; #1}}
\lhead[\fancyplain{}{\thepage}]{\fancyplain{}{\rightmark}}
\rhead[\fancyplain{}{\leftmark}]{\fancyplain{}{\thepage}}
\fancyfoot{}
\thispagestyle{fancy}%


\addtolength{\headwidth}{0cm}
\unitlength1mm %Define la unidad LE para Figuras
\mathindent0cm %Define la distancia de las formulas al texto,  fleqn las descentra
\marginparwidth0cm
\parindent0cm %Define la distancia de la primera linea de un parrafo a la margen

%Para tablas,  redefine el backschlash en tablas donde se define la posici\'{o}n del texto en las
%casillas (con \centering \raggedright o \raggedleft)
\newcommand{\PreserveBackslash}[1]{\let\temp=\\#1\let\\=\temp}
\let\PBS=\PreserveBackslash

%Espacio entre lineas
\renewcommand{\baselinestretch}{1.1}

%Neuer Befehl f\"{u}r die Tabelle Eigenschaften der Aktivkohlen
\newcommand{\arr}[1]{\raisebox{1.5ex}[0cm][0cm]{#1}}

%Neue Kommandos
\usepackage{Befehle}


%Trennungsliste
\hyphenation {Reaktor-ab-me-ssun-gen Gas-zu-sa-mmen-set-zung
Raum-gesch-win-dig-keit Durch-fluss Stick-stoff-gemisch
Ad-sorp-tions-tem-pe-ra-tur Klein-schmidt
Kohlen-stoff-Mole-kular-siebe Py-rolysat-aus-beu-te
Trans-port-vor-gan-ge}


\begin{document}

\section{Epidemiología Panorámica}

\justifying
Landscape Epidemiology [21, 22] promotes the notion that satellite data
from earth observation and geospatial technologies are essential tools [23] to
address vector borne epidemiological problems. Using these ideas, several
interdisciplinary studies were produced in latinoamerica focused in generat-
ing spatial and temporal predictive risk models based on satellite derived
environmental conditions [24, 25, 26, 27]. In particular in Argentine there
are some interesting experiences on this issue, [28, 29, 30] deal with Dengue
epidemics leading to operational tools [31]. At a global scope we can find
interesting contributions [32, 33, 34] with also some operatives experiences
[35]


El concepto interdisciplinario de \textit{Epidemiología Panorámica} (EP)
ha permitido generar algunos progresos en la epidemiología del dengue y otras
enfermedades transmitidas por vectores y zoonóticas como el chagas, la
malaria, la leishmaniasis y el hantavirus. Esta herramienta, EP, se centra en
producir mapas de riesgo predictivos espaciales y temporales basados en
características ambientales junto con datos de campo [7-14].
Estos estudios en Latinoamérica, donde se utiliza la tecnología espacial en
problemas epidemiológicos, están inspirados en trabajos pioneros llevados a cabo
en EE. UU. y Europa [15-18].
Con base en los trabajos citados en el párrafo anterior, en 2011, Argentina
comenzó a desarrollar un proyecto operacional
(Sistema de Alerta Temprana de Salud, HEWS), útil tanto para las autoridades de
salud como para los investigadores.

Básicamente, HEWS es un mapeo de riesgo
dinámico del dengue para todas las ciudades del país. En este producto, cada
ciudad es representada por un punto al que se le asigna un valor de riesgo para
cada año, basado en tecnología geoespacial. El trabajo fue realizado en un
contexto interdisciplinario e interinstitucional.

En este sistema [13], el riesgo se evalúa en cuatro componentes que son: el
entomológico, el viral, el componente relacionado con las actividades de
control y finalmente el ambiental. Mientras que los tres primeros componentes
se generan con el aporte de información de los agentes de salud que trabajan en cada
ciudad, el cuarto se evalúa a partir de datos satelitales.

Específicamente el componente ambiental, en la versión inicial del sistema, se
evalúa con una probabilidad estacionaria de presencia de vectores (igual para
todo el tiempo) más un componente relacionado con el número de ciclos virales,
que son una función de la temperatura, y así es diferente para cada ciudad y
para cada año. El mapa de probabilidad de presencia de especie (modelo de nicho)
es claramente una gran simplificación y se puede mejorar en base a datos
satelitales continuos del medio ambiente. Variables como precipitación y
temperatura, han demostrado, con una variabilidad local, influenciar el
desarrollo de mosquitos, su supervivencia y actividad de oviposición y por ende
la abundancia de vectores.

En particular trabajos previos como los de Estallo [8], [19] utilizan información
satelital para estimar la evolución temporal de la abundancia de vectores.
En 2017 German y colaboradores [20] desarrollan una metodología completa
para generar modelos de manera automática y basada en información de libre
acceso. En particular German [20] utiliza productos del sensor (MODIS) a bordo
del satélite Terra y Aqua, pues es uno de los más adecuados para esta
aplicación particular, debido a su resolución temporal, espectral y espacial.
MODIS proporciona un conjunto de productos pre-procesados y de libre acceso [21].
Específicamente, los productos de vegetación (índice de vegetación de diferencia
normalizada) y temperatura (temperatura de la superficie terrestre) derivados
de MODIS son ejemplos de variables de percepción remota utilizadas
en aplicaciones de epidemiologia [13], [22] incorporadas en [20]. Otra variable
ambiental obtenida de satélite que es relevante e incorporada, es el
\textit{Índice de Agua de Diferencia Normalizada} (NDWI) que evalúa de alguna
forma el contenido de agua de la cubierta. Adicionalmente el trabajo de German
incorpora una estimación de la precipitación desde el espacio a partir de las
misiones (TRMM) y (GPM) [23].



-----------------
Landscape epidemiology: An emerging perspective in the mapping and
modelling of disease and disease risk factors

Remote sensing ( RS ) enables scientists to study the biotic
and abiotic components of the earth surface. RS essentially
measures energy reflected or emitted in distinct and
specific electromagnetic spectrum, using sensors usually
onboard satellites. T hey are used to monitor and observe the
earth’s landscape. T he epidemiological applications of RS
and GIS have been reviewed extensively

S ince the launch of ERTS - 1 ( E arth R esources T echnology
S atellites ) in 1972 , N ational A eronautics and S pace
A dministration ( NASA ’s ) C enter for H ealth A pplications



A erospace R elated T echnologies ( CHAART ) has initiated
programs aimed at integrating these technologies into the
areas of forestry, agriculture, geology and public health.
T he worsening health conditions around the world and
the significant advancement in computer processing,
improvement in data acquisition, reduced hardware and
software cost and the development of computer based GIS
technology have led to the launching of programs that aim at
integrating RS / GIS into health applications by CHAART .
T o better apply these technologies and satellite-sensor
capabilities, NASA - CHAART developed a website to
evaluate sensors for health applications. T hey defined
16 groups of physical factors that could be used for both
research and in practice. E ach of these factors is essentially
an environmental variable that might have a direct and
indirect bearing in the survival of pathogen, vectors,
reservoir or host. T hese factors ( biotic and abiotic ) may
also affect many types of non-vector borne diseases such
as water borne diseases. E xamples of these factors include
soil moisture, wetlands, vegetation and crop type and sea
surface temperature. A ll these factors can be sensed by a
sensor mounted on a satellite imagery used for landscape
epidemiology. S ome of the links between these factors and
various diseases as developed by CHAART are listed in

T able 1 .
T wo approaches used for mapping diseases are the
correlation of host distribution with climatic data or
correlation with landscape. M ost human studies using RS
have focused on data obtained from L andsat’s M ultispectral
S canner ( MSS ) and T hematic M apper ( TM ) , the N ational
O ceanic and A tmospheric A dministration ( NOAA ) ’s
A dvanced V ery H igh R esolution R adiometer ( AVHRR ) , and
F rance’s S ystème P our l’ O bservation de la T erre ( SPOT ) .
I n many of these studies, remotely sensed data were used
to derive three variables: landscape structure, vegetation
cover and water bodies. T able 2 contains examples some
researches conducted using remote sensing.












\end{document}
