\documentclass[12pt,spanish,fleqn,openany,letterpaper,pagesize]{scrbook}

\usepackage[utf8]{inputenc}
\usepackage[spanish]{babel}
\usepackage{fancyhdr}
\usepackage{epsfig}
\usepackage{epic}
\usepackage{eepic}
\usepackage{amsmath}
\usepackage{threeparttable}
\usepackage{amscd}
\usepackage{here}
\usepackage{graphicx}
\usepackage{lscape}
\usepackage{tabularx}
\usepackage{subfigure}
\usepackage{longtable}


\usepackage{rotating} %Para rotar texto, objetos y tablas seite. No se ve en DVI solo en PS. Seite 328 Hundebuch
                        %se usa junto con \rotate, \sidewidestable ....


\renewcommand{\theequation}{\thechapter-\arabic{equation}}
\renewcommand{\thefigure}{\textbf{\thechapter-\arabic{figure}}}
\renewcommand{\thetable}{\textbf{\thechapter-\arabic{table}}}


\pagestyle{fancyplain}%\addtolength{\headwidth}{\marginparwidth}
\textheight22.5cm \topmargin0cm \textwidth16.5cm
\oddsidemargin0.5cm \evensidemargin-0.5cm%
\renewcommand{\chaptermark}[1]{\markboth{\thechapter\; #1}{}}
\renewcommand{\sectionmark}[1]{\markright{\thesection\; #1}}
\lhead[\fancyplain{}{\thepage}]{\fancyplain{}{\rightmark}}
\rhead[\fancyplain{}{\leftmark}]{\fancyplain{}{\thepage}}
\fancyfoot{}
\thispagestyle{fancy}%


\addtolength{\headwidth}{0cm}
\unitlength1mm %Define la unidad LE para Figuras
\mathindent0cm %Define la distancia de las formulas al texto,  fleqn las descentra
\marginparwidth0cm
\parindent0cm %Define la distancia de la primera linea de un parrafo a la margen

%Para tablas,  redefine el backschlash en tablas donde se define la posici\'{o}n del texto en las
%casillas (con \centering \raggedright o \raggedleft)
\newcommand{\PreserveBackslash}[1]{\let\temp=\\#1\let\\=\temp}
\let\PBS=\PreserveBackslash

%Espacio entre lineas
\renewcommand{\baselinestretch}{1.1}

%Neuer Befehl f\"{u}r die Tabelle Eigenschaften der Aktivkohlen
\newcommand{\arr}[1]{\raisebox{1.5ex}[0cm][0cm]{#1}}

%Neue Kommandos
\usepackage{Befehle}


%Trennungsliste
\hyphenation {Reaktor-ab-me-ssun-gen Gas-zu-sa-mmen-set-zung
Raum-gesch-win-dig-keit Durch-fluss Stick-stoff-gemisch
Ad-sorp-tions-tem-pe-ra-tur Klein-schmidt
Kohlen-stoff-Mole-kular-siebe Py-rolysat-aus-beu-te
Trans-port-vor-gan-ge}


\begin{document}

\subsection{Machine Learning}


El Aprendizaje Automático (ML) es un enfoque empírico efectivo para
regresiones y/o clasificaciones de sistemas no-lineales que pueden involucrar desde
unos pocos hasta varios cientos de variables. El enfoque de ML requiere
entrenamiento utilizando un conjunto de datos que sea representativo del conjunto
de datos del problema. Se debe seleccionar un conjunto de datos que se utilizará,
luego del entrenamiento del modelo, para validación del mismo.
ML es ideal para aquellos problemas en donde el conocimiento teórico del mismo
es incompleto o insuficiente, pero se cuenta con un gran conjunto de observaciones.
ML ha mostrado ser de utilidad para un gran número de aplicaciones en Geociencias
relacionadas a la tierra, oceanos y atmósfera, y en algoritmos de extracción
de información bio-geofísica.
Algunos de los algoritmos de ML más usados en aplicaciones relativas a
Geociencias y Sensado Remoto (GRS) son las Redes Neuronales Artificiales (ANN),
Support Vector Machines (SVM), Mapas Auto-organizados (SOM), Árboles de Decisión (DT),
Random Forests y Algoritmos Genéticos. Su aplicación en problemas de GRS es
relativamente nuevo y extremádamente prometedora. En particular, ANNs son
usadas para clasifición pero también son usadas para la aplicación en pronósticos
relativos a series de tiempo.
De hecho, una exploración en la base bibliográfica Scopus devuelve más de 4000
publicaciones que incluyen \textit{remote sensing} y \textit{neural network},
311 de ellas en 2016. De ese total, el 45\% corresponde a
\textit{Sciences of the Earth}, el 44\% a \textit{Computer Science} y el 35\% a
\textit{Engineering}; con China, Estados Unidos, Italia e India como los paises
con mayor producción científica en dichas áreas.

\end{document}
