%\newpage
%\setcounter{page}{1}
\begin{center}
\begin{figure}
\centering%
\epsfig{file=HojaTitulo/logoFAMAF.jpg,scale=0.8}%
\end{figure}
\thispagestyle{empty} \vspace*{0.1cm} \textbf{\huge
Estratificación temporal de Aedes Aegypti basada en herramientas geoespaciales y aprendizaje automático}\\[9.0cm]
\Large\textbf{Juan Manuel Scavuzzo}\\[0.5cm]
\small Universidad Nacional de Córdoba\\
Facultad de Matemática, Astronomía, Física y Computación\\
Córdoba, Argentina\\
2018\\
\end{center}

\newpage{\pagestyle{empty}\cleardoublepage}

\newpage
\begin{center}
\thispagestyle{empty} \vspace*{0cm} \textbf{\huge
Estratificación temporal de Aedes Aegypti basada en herramientas Geoespaciales y Machine Learning}\\[3.0cm]
\Large\textbf{Juan Manuel Scavuzzo}\\[2.0cm]
\small Tesis de grado presentada como requisito parcial para optar al
título de:\\
\textbf{Licenciado en Ciencias de la Computación}\\[2.5cm]
Directores:\\

Mgter. Gonzalo Sebastián Peralta (Licenciado en Cs de la Computación y Magíster en Aplicaciones Espaciales)\\
Dr. Jorge Sánchez (Ingeniero en Electrónica y Doctor en Ciencias de la Ingeniería, Visión por computadoras y reconocimiento de patrones)\\ [3.0cm]

Universidad Nacional de Córdoba\\
Facultad de Matemática, Astronomía, Física y Computación\\
Córdoba, Argentina\\
2018\\\end{center}

\newpage{\pagestyle{empty}\cleardoublepage}

\newpage
\thispagestyle{empty} \textbf{}\normalsize
\\\\\\%
\textbf{(Dedicatoria o un lema)}\\[4.0cm]

\begin{flushright}
\begin{minipage}{8cm}
    \noindent
        \small
        Aca va algun lema o algo asi
        Por ejemplo:\\[1.0cm]
        A mis padres\\[1.0cm]\\
        o\\[1.0cm]
        La preocupaci\'{o}n por el hombre y su destino siempre debe ser el
        inter\'{e}s primordial de todo esfuerzo t\'{e}cnico. Nunca olvides esto
        entre tus diagramas y ecuaciones.\\\\
        Albert Einstein\\
\end{minipage}
\end{flushright}

\newpage{\pagestyle{empty}\cleardoublepage}

\newpage
\thispagestyle{empty} \textbf{}\normalsize
\\\\\\%
\textbf{\LARGE Agradecimientos}
\addcontentsline{toc}{chapter}{\numberline{}Agradecimientos}\\\\
insertar agradecimientos

\newpage{\pagestyle{empty}\cleardoublepage}

\newpage
\textbf{\LARGE Resumen}
\addcontentsline{toc}{chapter}{\numberline{}Resumen}\\\\

\textbf{\small Palabras clave: Computer Science, Machine Learning, Python,
      Landscape Epidemiology, Remote Sensing, Dengue, Zika, Chikungunya, Public Health}.\\
\justifying
  \par El Dengue, Zika y Chikungunya son enfermedades virales cuya vacuna para
    prevención aún no existe y que, en los últimos años, han tenido un incremento
    e impacto en la población de la región argentina y latinoamericana que ha
    sido de gran preocupación para los organismos gubernamentales de salud.

  \par En los últimos años se han generado sistemas de riesgo de transmisión
    de enfermedades virales basados en información de sensores remotos,
    estableciendo relaciones entre las condiciones ambientales de las distintas
    zonas con la abundancia del vector en las mismas. A dicha área de estudio
    se la denomina Epidemiología Panorámica.

 \par En el presente trabajo, por un lado, utilizando técnicas de
    ingeniería del software para extraer los requerimientos y aplicar una metodología
    de desarrollo acorde a las necesidades,
    se implementa un \textit{framework} para la generación de modelos de
    aprendizaje automático (ML) con el objetivo
    de estimar la abundancia de vectores de Dengue, Zika y Chikungunya.
    A su vez, se entrenan y evalúan modelos no lineales que poseen mayor
    capacidad de generalización a la hora de modelar las poblaciones del
    mosquito, en comparación con los modelos que actualmente se utilizan
    para tal fin.

  \par Se presenta, además, un enfoque que resuelve el problema de que
    un modelo entrenado con información de una sola ciudad no es capaz,
    en principio, de estimar correctamente la abudancia en otras zonas del país.
    En este trabajo
    se propone resolver la cuestión a traves de un concepto novedoso en el campo
    de la epidemiología, que establece
    relaciones de ``cercanía``
    entre regiones teniendo en cuenta sus características
    ambientales: la Distancia Ambiental Normalizada.
