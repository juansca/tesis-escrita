%\newpage
%\setcounter{page}{1}
\begin{center}
\begin{figure}
\centering%
\epsfig{file=HojaTitulo/logoFAMAF.jpg,scale=0.8}%
\end{figure}
\thispagestyle{empty} \vspace*{0.1cm} \textbf{\huge
Estratificación temporal de Aedes Aegypti basada en herramientas geoespaciales y aprendizaje automático}\\[8.5cm]
\Large\textbf{Juan Manuel Scavuzzo}\\[0.5cm]
\small Universidad Nacional de Córdoba\\
Facultad de Matemática, Astronomía, Física y Computación\\
Córdoba, Argentina\\
2018\\
\end{center}

\newpage{\pagestyle{empty}\cleardoublepage}

\newpage
\begin{center}
\thispagestyle{empty} \vspace*{0cm} \textbf{\huge
Estratificación temporal de Aedes Aegypti basada en herramientas geoespaciales y aprendizaje automático}\\[6.0cm]
\Large\textbf{Juan Manuel Scavuzzo}\\[2.0cm]
\small Trabajo final de grado presentado como requisito parcial para optar al
título de:\\
\textbf{Licenciado en Ciencias de la Computación}\\[2.0cm]
Directores:\\

Dr. Jorge Sánchez (Ingeniero en Electrónica y Doctor en Ciencias de la Ingeniería, Visión por computadoras y reconocimiento de patrones)\\
Mgter. Gonzalo Sebastián Peralta (Licenciado en Cs de la Computación y Magíster en Aplicaciones Espaciales)\\ [2.0cm]

Universidad Nacional de Córdoba\\
Facultad de Matemática, Astronomía, Física y Computación\\
Córdoba, Argentina\\
2018\\\end{center}


\newpage{\pagestyle{empty}\cleardoublepage}

\newpage
\thispagestyle{empty} \textbf{}\normalsize
\\\\\\%
\textbf{\LARGE Agradecimientos}
\addcontentsline{toc}{chapter}{\numberline{}Agradecimientos}\\\\
\justifying

  \par Soy de las personas que piensan que los logros son colectivos. Por más que
    los títulos estén a nombre de una sola persona, no sería posible realizar logros
    plenamente individuales, y si lo fuera, éstos no serían lo mismo... tendrían
    vacíos, les faltaría algo. Los logros colectivos nos permiten entender y aprender
    de manera integral. Nos enseñan a que siempre tenemos algo que aprender del otro
    y tenemos algo que convidar para el otro... Para mí, un buen profesional es
    aquel que comprende que tiene algo que aportar a la sociedad
    (y muchas cosas que aprender de ella),
    ya sea desde lo técnico o desde la capacidad adquirida para razonar sobre
    cuestiones cotidianas.

  \par Es por todas las razones que menciono, que agradezco plenamente a todas las
    personas que pasaron, durante todos estos años, por mi vida. Nombrar uno
    por uno los nombres, quizás llevaría demasiado y por eso me toca hacer
    énfasis y mencionar a aquellas que estuvion más cerca durante todo
    este proceso.

  \par Quiero agradecer a mis viejos, Trinidad y Marcelo, quienes con todo su esfuerzo
    me brindaron la posibilidad de dedicarme a estudiar y enfocar mis esfuerzos en
    aquello que yo elegí como proyecto de vida, siempre apoyándome en cualquier
    decisión tomada.

  \par A Lula, mi compañera de vida, mi cómplice y mejor amiga. Quién me apoyó,
    aguantó y disfrutó los últimos años de este gran camino recorrido. Sin
    dudas, fue de las personas que me llenó de fuerzas en los últimos tramos, cuando
    la energía escasea.

  \par A mis hermanos, Marco y Matías, que desde siempre estuvieron ahí
    acompañando y apoyando por más que también estuvieran haciendo su
    camino al andar, con todas las dificultades que eso implica.

  \par A Gonzalo, quien desde su lugar cercano a la familia, es uno de los
  responsables de que haya elegido esta carrera. Quien, a pesar del cariño,
  fue mi docente y director de tesis pudiendo exigirme para verme crecer.

  \par A mis amigos-compañeros hechos en las clases, Trucco, Agus, Fran, Marcos
  y Limón con quienes pasamos muchas andanzas y horas de estudio. Sin los
  cuales, hubiera sido de gran dificultad afrontar todas las dificultades
  de la carrera. Son gente que
  me llevo para las próximas etapas que toque vivir.


\newpage{\pagestyle{empty}\cleardoublepage}

\newpage
\textbf{\LARGE Resumen}
\addcontentsline{toc}{chapter}{\numberline{}Resumen}\\\\

\textbf{\small Palabras clave: Computer Science, Machine Learning, Python,
      Landscape Epidemiology, Remote Sensing, Dengue, Zika, Chikungunya, Public Health}.\\
\justifying
  \par El Dengue, Zika y Chikungunya son enfermedades virales cuya vacuna para
    prevención aún no existe y que, en los últimos años, han tenido un incremento
    e impacto en la población de la región argentina y latinoamericana. Razón
    por la cual, son
    una gran preocupación para los organismos gubernamentales de salud.

  \par En los últimos años se han generado sistemas para la estimación de riesgo de transmisión
    de enfermedades virales basados en información de sensores remotos,
    estableciendo relaciones entre las condiciones ambientales de las distintas
    zonas y la abundancia del vector en las mismas. A dicha área de estudio
    se la denomina Epidemiología Panorámica.

 \par En el presente trabajo, por un lado, se utilizan técnicas de
    ingeniería del software para extraer los requerimientos y aplicar una metodología
    de desarrollo acorde a las necesidades,
    para implementar un \textit{framework} para la generación de modelos de
    aprendizaje automático (ML) con el objetivo
    de estimar la abundancia de vectores de Dengue, Zika y Chikungunya.
    A su vez, se entrenan y evalúan modelos no lineales para modelar las poblaciones del
    mosquito. Éstos poseen mayor
    capacidad de generalización, en comparación con los modelos que
    actualmente se utilizan para tal fin.

  \par Se presenta, además, un enfoque que resuelve el problema de que
    un modelo entrenado con información de una sola ciudad no es capaz,
    en principio, de estimar correctamente la abudancia en otras zonas del país.
    En este trabajo
    se propone resolver la cuestión a traves de un concepto novedoso en el campo
    de la epidemiología, que establece
    relaciones de ``cercanía``
    entre regiones teniendo en cuenta sus características
    ambientales: la Distancia Ambiental Normalizada.
