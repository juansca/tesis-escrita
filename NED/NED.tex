\documentclass[12pt,spanish,fleqn,openany,letterpaper,pagesize]{scrbook}

\usepackage[utf8]{inputenc}
\usepackage[spanish]{babel}
\usepackage{fancyhdr}
\usepackage{epsfig}
\usepackage{epic}
\usepackage{eepic}
\usepackage{amsmath}
\usepackage{amssymb}
\usepackage{siunitx}
\usepackage{booktabs}
\usepackage{lineno}
\usepackage{threeparttable}
\usepackage{amscd}
\usepackage{here}
\usepackage{graphicx}
\usepackage{lscape}
\usepackage{tabularx}
\usepackage{url}
\usepackage{subfigure}
\usepackage{longtable}
\usepackage{framed}
\usepackage{ragged2e} \justifying
\usepackage{rotating} %Para rotar texto, objetos y tablas seite. No se ve en DVI solo en PS. Seite 328 Hundebuch
                        %se usa junto con \rotate, \sidewidestable ....
\usepackage{tikz}
\usepackage{verbatim}


\usepackage{bbm}

\usepackage{listings}
\usepackage{color}

\definecolor{dkgreen}{rgb}{0,0.6,0}
\definecolor{gray}{rgb}{0.5,0.5,0.5}
\definecolor{mauve}{rgb}{0.58,0,0.82}

\lstset{frame=tb,
  language=Bash,
  aboveskip=3mm,
  belowskip=3mm,
  showstringspaces=false,
  columns=flexible,
  basicstyle={\small\ttfamily},
  numbers=none,
  numberstyle=\tiny\color{gray},
  keywordstyle=\color{blue},
  commentstyle=\color{dkgreen},
  stringstyle=\color{mauve},
  breaklines=true,
  breakatwhitespace=true,
  tabsize=3
}

\renewcommand{\theequation}{\thechapter-\arabic{equation}}
\renewcommand{\thefigure}{\textbf{\thechapter-\arabic{figure}}}
\renewcommand{\thetable}{\textbf{\thechapter-\arabic{table}}}
\newcommand{\norm}[1]{\left\lVert#1\right\rVert}

\pagestyle{fancyplain}%\addtolength{\headwidth}{\marginparwidth}
\textheight22.5cm \topmargin0cm \textwidth16.5cm
\oddsidemargin0.5cm \evensidemargin-0.5cm%
\renewcommand{\chaptermark}[1]{\markboth{\thechapter\; #1}{}}
\renewcommand{\sectionmark}[1]{\markright{\thesection\; #1}}
\lhead[\fancyplain{}{\thepage}]{\fancyplain{}{\rightmark}}
\rhead[\fancyplain{}{\leftmark}]{\fancyplain{}{\thepage}}
\fancyfoot{}
\thispagestyle{fancy}%


\addtolength{\headwidth}{0cm}
\unitlength1mm %Define la unidad LE para Figuras
\mathindent0cm %Define la distancia de las formulas al texto,  fleqn las descentra
\marginparwidth0cm
\parindent0cm %Define la distancia de la primera linea de un parrafo a la margen

%Para tablas,  redefine el backschlash en tablas donde se define la posici\'{o}n del texto en las
%casillas (con \centering \raggedright o \raggedleft)
\newcommand{\PreserveBackslash}[1]{\let\temp=\\#1\let\\=\temp}
\let\PBS=\PreserveBackslash

%Espacio entre lineas
\renewcommand{\baselinestretch}{1.1}

%Neuer Befehl f\"{u}r die Tabelle Eigenschaften der Aktivkohlen
\newcommand{\arr}[1]{\raisebox{1.5ex}[0cm][0cm]{#1}}

%Neue Kommandos
\usepackage{Befehle}


%Trennungsliste
\hyphenation {Reaktor-ab-me-ssun-gen Gas-zu-sa-mmen-set-zung
Raum-gesch-win-dig-keit Durch-fluss Stick-stoff-gemisch
Ad-sorp-tions-tem-pe-ra-tur Klein-schmidt
Kohlen-stoff-Mole-kular-siebe Py-rolysat-aus-beu-te
Trans-port-vor-gan-ge}

\begin{document}

\justifying

\chapter{Generalización espacial de modelos epidemiológicos basada en el
        concepto de Distancia Ambiental Normalizada NED}

  \par Los modelos temporales descriptos en capitulos anteriores se basan en la
    generación de relaciones empíricas entre datos ambientales derivados de
    información satelital y los datos de campo, correspondientes a los del vector
    propiamente dicho. Esto significa que sólo pueden construirse modelos en
    lugares donde esté disponible la información de campo, problema que mencionamos
    al concluir el capitulo anterior.

  \par En ese marco, y con el objetivo final de mejorar la aplicación operativa
    presentada por Porcasi y colaboradores en 2012 \cite{porcasi_operative},
    nos planteamos en este capítulo el objetivo específico de generar
    una metodología para espacializar los datos contruidos siguiendo la
    metodología de \cite{german_temporal}, basada en el concepto de
    \textbf{\textit{Distancia Ambiental Normalizada}} (NED).


\section{Materiales y Métodos}

\subsection{Problema}

  \par A partir de la disponibilidad de datos de campo en $N$ localidades
    diferentes, se generan, siguiendo la metodología presentada en \cite{german_temporal}
    $N$ modelos diferentes que relacionan la oviposición a variables ambientales
    derivadas de datos satelitales (\textit{lst\_night}, \textit{lst\_day},
    \textit{ndvi}, \textit{ndwi}, \textit{prec}). Por simplicidad, sin pérdida
    de generalidad, supongamos que dichos modelos son lineales, por lo tanto
    son de la forma:
    \begin{align}
      ovip_{j} = \beta_{j} + \sum{}{coef_{ji} \times envVar_{i}(j)}
    \end{align}

  \par Donde $coef_{ji}$ representa los coeficientes del modelo de la
    ciudad $j$ para la variable $i$, y $envVar_{i}(j)$ representa la variable
    ambiental $i$ evaluada en la posición correspondiente a la ciudad $j$.
    Es decir que para cada ciudad $j$, hay un conjunto diferente de
    coeficientes, que son aquellos que generan un ajuste óptimo de los datos
    disponibles. Aquí denominamos a estos $N$ modelos $M_{1},\ M_{2},\ \dots,\ M_{N}$.


  \par Así, el problema que se plantea es aquel en el que el modelo se
    debiera utilizar en una nueva ciudad (no incluida en las $N$ anteriores),
    para obtener una estimación de la abundancia del vector, para cualquier otra
    ciudad. En particular en este caso, en la región norte de Argentina donde
    no se disponga de datos de campo.


  \par La idea más simple para extrapolar los modelos obtenidos, sería usar,
    para un punto/pueblo adicional localizado en la posición $x$, un modelo $M_{X}$
    igual al modelo conocido de la ciudad más cercana geográficamente (vecino más cercano) es
    decir $M_{X}\ =\ M_{J}$ donde $J$ corresponde a la ciudad más cercana.
    Una mejora a este enfoque, es utilizar un promedio de los $N$ modelos conocidos
    ponderados por el inverso de la distancia de este nuevo punto $X$ a cada una
    de las ciudades $J$ donde se dispone de un modelo. Es decir, el modelo de la
    ciudad más cercana pesará más y el de la más alejada pesará menos, es decir:

    \begin{align}
      M_{X} = \sum{}{\frac{M_{j}}{L_{j}}}
    \end{align}

    donde $L_{j}$ representa la distancia normalizada de la ciudad $J$ a
    $X$ (la localización geográfica de la nueva ciudad).


  \par El problema de las soluciones anteriores, es que realmente es más
    razonable pensar que el comportamiento de la población del vector/mosquito
    en una ciudad en el punto $X$ será más coincidente con una que se encuentre
    en una ciudad que sea más similar \textbf{ambientalmente} y no necesariemente
    con aquella que está más cerca geográficamente. En ese sentido, se debería
    utilizar (en el esquema de vecino más cercano) el modelo de la ciudad $J$
    que posea medio ambiente más similar al del punto $X$. En otras palabras,
    asi como ``más cerca", significa típicamente coordenadas geográficas (o posiciones)
    similares; en el sentido ecológico/ambiental, podemos pensar ``más cerca"
    como que sus variables ambientales son similares. De esta forma aparece
    naturalmente el concepro de \textbf{\textit{Distancia Ambiental}}.

\subsection{Distancia Ambiental Normalizada (NED)}


\end{document}
