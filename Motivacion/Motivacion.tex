\documentclass[12pt,spanish,fleqn,openany,letterpaper,pagesize]{scrbook}

\usepackage[utf8]{inputenc}
\usepackage[spanish]{babel}
\usepackage{fancyhdr}
\usepackage{epsfig}
\usepackage{epic}
\usepackage{eepic}
\usepackage{amsmath}
\usepackage{amssymb}
\usepackage{siunitx}
\usepackage{booktabs}
\usepackage{lineno}
\usepackage{threeparttable}
\usepackage{amscd}
\usepackage{here}
\usepackage{graphicx}
\usepackage{lscape}
\usepackage{tabularx}
\usepackage{url}
\usepackage{subfigure}
\usepackage{longtable}
\usepackage{framed}
\usepackage{ragged2e} \justifying
\usepackage{rotating} %Para rotar texto, objetos y tablas seite. No se ve en DVI solo en PS. Seite 328 Hundebuch
                        %se usa junto con \rotate, \sidewidestable ....
\usepackage{tikz}
\usepackage{verbatim}


\usepackage{bbm}

\usepackage{listings}
\usepackage{color}

\definecolor{dkgreen}{rgb}{0,0.6,0}
\definecolor{gray}{rgb}{0.5,0.5,0.5}
\definecolor{mauve}{rgb}{0.58,0,0.82}

\lstset{frame=tb,
  language=Bash,
  aboveskip=3mm,
  belowskip=3mm,
  showstringspaces=false,
  columns=flexible,
  basicstyle={\small\ttfamily},
  numbers=none,
  numberstyle=\tiny\color{gray},
  keywordstyle=\color{blue},
  commentstyle=\color{dkgreen},
  stringstyle=\color{mauve},
  breaklines=true,
  breakatwhitespace=true,
  tabsize=3
}

\renewcommand{\theequation}{\thechapter-\arabic{equation}}
\renewcommand{\thefigure}{\textbf{\thechapter-\arabic{figure}}}
\renewcommand{\thetable}{\textbf{\thechapter-\arabic{table}}}
\newcommand{\norm}[1]{\left\lVert#1\right\rVert}

\pagestyle{fancyplain}%\addtolength{\headwidth}{\marginparwidth}
\textheight22.5cm \topmargin0cm \textwidth16.5cm
\oddsidemargin0.5cm \evensidemargin-0.5cm%
\renewcommand{\chaptermark}[1]{\markboth{\thechapter\; #1}{}}
\renewcommand{\sectionmark}[1]{\markright{\thesection\; #1}}
\lhead[\fancyplain{}{\thepage}]{\fancyplain{}{\rightmark}}
\rhead[\fancyplain{}{\leftmark}]{\fancyplain{}{\thepage}}
\fancyfoot{}
\thispagestyle{fancy}%


\addtolength{\headwidth}{0cm}
\unitlength1mm %Define la unidad LE para Figuras
\mathindent0cm %Define la distancia de las formulas al texto,  fleqn las descentra
\marginparwidth0cm
\parindent0cm %Define la distancia de la primera linea de un parrafo a la margen

%Para tablas,  redefine el backschlash en tablas donde se define la posici\'{o}n del texto en las
%casillas (con \centering \raggedright o \raggedleft)
\newcommand{\PreserveBackslash}[1]{\let\temp=\\#1\let\\=\temp}
\let\PBS=\PreserveBackslash

%Espacio entre lineas
\renewcommand{\baselinestretch}{1.1}

%Neuer Befehl f\"{u}r die Tabelle Eigenschaften der Aktivkohlen
\newcommand{\arr}[1]{\raisebox{1.5ex}[0cm][0cm]{#1}}

%Neue Kommandos
\usepackage{Befehle}


%Trennungsliste
\hyphenation {Reaktor-ab-me-ssun-gen Gas-zu-sa-mmen-set-zung
Raum-gesch-win-dig-keit Durch-fluss Stick-stoff-gemisch
Ad-sorp-tions-tem-pe-ra-tur Klein-schmidt
Kohlen-stoff-Mole-kular-siebe Py-rolysat-aus-beu-te
Trans-port-vor-gan-ge}


\begin{document}

\chapter{Motivación}

\justifying
\par El mosquito es uno de los vectores de enfermedades humanas más importantes
  en el mundo. En particular, el \textit{Aedes aegypti} es el principal vector
  de Dengue, Chikungunya, Zika y Fiebre Amarilla urbana [1].
  Según datos de la Organización Mundial de la Salud (OMS), alrededor de 80 millones de
  personas se infectan de Dengue anualmente, cerca de 550 mil enfermos requieren hospitalización y
  unos 20 mil mueren. Además, calculan que más de 2.500 millones de personas corren
  riesgo de contraer la enfermedad y más de 100 países tienen transmisión endémica.
  [Directrices para la prevención y control de Aedes aegypti, está el pdf]
  Algo que cabe aclarar es que en el caso de la Fiebre Amarilla, para la
  prevencion, existe una vacuna de virus atenuado que se considera eficaz, segura
  y se la utiliza hace más de 60 años para la inmunización activa de niños y
  adultos. No es así el caso del Dengue,
  Chikungunya y Zika, para las cuales no existe tal herramienta de previsión.

\par Si tenemos en cuenta las 4 enfermedades mencionadas en el parrafo anterior,
  en las Américas, entre 1985 y 2012, el 95\% de los casos se concentraron en
  4 países: Perú (54\% de los casos), Bolivia (18\%), Brasil (16\%) y Colombia (7\%).
  Otras naciones de la región que presentan condiciones muy
  favorables para la transimisión son Argentina, Ecuador, Panamá y Venezuela.
  Desde 2000 a 2013, más de 1.100 casos confirmados por laboratorio fueron
  reportados.
  [http://www.mdm.org.ar/prensa/articulo/221/Mdicos-del-Mundo-alerta-sobre-riesgos-de-fiebre-amarilla-en-Brasil-y-escenarios-de-Dengue-Zika-en-Argentina\#.W1yUa6zV-AU.link]

\par En el caso de Argentina, ya para las primeras semanas del 2018, hubo casos confirmados
  de Dengue en Chaco, y durante el 2017, en base a las notificaciones al
  \textbf{Sistema Nacional de Vigilancia de Salud} del Ministerio de Salud de la Nación
  recibidas hasta el 30 de diciembre, se registraron, en el primer semestre del año, brotes de
  Dengue serotipo DEN-1 con 646 casos confirmados en 5 provincias
  (Buenos Aires, Chaco, Corrientes, Formosa y Santa Fe) y 253 casos de enfermedad
  por virus del Zika en 3 provincias (Chaco, Formosa y Salta).
  Desde la emergencia del virus del Zika en nuestro país en el 2016 (Tucumán), y hasta
  la [SE 47 REVISAR] de 2017 se registraron además un total de 7 casos confirmados de
  síndrome congénito asociados a virus del Zika en mujeres embarazadas
  (microcefalia en recién nacidos).

\par El Dr. Gonzalo Basile \footnote{Presidente Honor y Director General para
                 América Latina y Caribe de Médicos del Mundo, e investigador de institutos de
                 investigación en salud pública del Caribe y coordinación regional del Programa
                 de Salud Internacional de CLACSO y de FLACSO República Dominicana}
     se refiere al incremento del riesgo de crecimiento
     en la cantidad de casos positivos en nuestro país, teniendo en cuenta
     el contexto epidemiológico en la región:
\begin{framed}

  Aunque las últimas epidemias del 2009 y 2016 de Dengue en Argentina fueron del
  serotipo DEN1, la circulación viral de los otros serotipos en la región de
  Cono Sur (Brasil, Paraguay y Bolivia) tanto DEN4, DEN2 y DEN3, hace que los
  periodos epidémicos de DEN se puedan modificar. Por otro lado, el escenario de
  Zika Virus es una realidad por su circulación en América Latina y Caribe con
  cuadros clínicos inéspecíficos pero con eventos asociados como el Síndrome de
  Guillaen Barré y microcefalia que implican problemas epidemiológicos
  poblacionales de incidencia como lo demostraron en Brasil, Colombia, Venezuela,
  República Dominicana, entre otros 47 países de la región donde se confirmaron
  casos de transmisión activa vectorial de Zika.
  Si sumamos ahora el brote epidémico de Fiebre Amarilla en Brasil con la
  posibilidad de reintroducir casos en el Cono Sur ya que las tasas de
  inmunizaciones para fiebre amarrilla existen brechas en varias ciudades de nuestro país \\

 \centering 24/01/2018
\end{framed}



\par Sumado a lo comentado, por su parte, el \textit{Aedes aegypti} se
  caracteriza por su presencia en el medio urbano, su preferencia
  de cría en contenedores artificiales [3] y la resistencia de sus huevos a la
  desecación. En nuestro país, además, los vertiginosos cambios demográficos, han dado
  por resultado una gran ampliación desorganizada de las zonas urbanas. Ésto, junto
  con el aumento del uso de recipientes no biodegradables y un método deficiente
  de recolección de residuos sólidos, incrementan el número de depósitos que
  acumulan agua, que actúan como potenciales criaderos del mosquito, lo cual aumenta el
  riesgo de ocurrencia de casos de las enfermedades mencionadas.
  Dado que la cantidad de vectores, el virus circulante y la susceptibilidad
  humana dependen directa o indirectamente de variables climáticas y ambientales [buscar cita]
  tales como la temperatura, la lluvia, la vegetación, entre otras, es razonable
  suponer que el cambio climático es, también, un factor de riesgo para el desarrollo
  de las enfermedades en cuestión. Por otra parte, se le suma la capacidad adaptativa del
  \textit{Aedes aegypti} y la aparición de resistencia del mismo debido al uso intensivo de
  insecticidas.


\par Por otro lado, como mencionamos anteriormente, dado que no hay vacunas para la
  mayoría de estos virus, y existe la posibilidad de introducción de otros [2],
  el control de vectores es la principal herramienta para mitigar la
  propagación de enfermedades.


\par Es claro que el escenario epidémico planteado es una realidad en Argentina
  que hay que atacar. Ésto deriva en la necesidad de enfocar esfuerzos en el
  desarrollo de estrategias contundentes dirigidas a evitar, limitar y controlar
  las poblaciones de \textit{Aedes aegyti}, lo que implica repensar y diseñar
  nuevos sistemas de alerta temprana, vigilancia epidemiológica y respuesta
  rápida desde lo local, integrando un espacio interinstitutional e
  intersectorial de coordinación, planificación e intervención pública. Ésto debe
  llevarse a cabo entre el Estado Nacional, municipios, universidades y centros
  de estudio, organizaciones civiles, entre otros actores sociales de gran importancia.
  En ese contexto, la introducción de herramientas científico/tecnológicas orientadas
  a contribuir en esos aspectos resulta fundamental.

\par El uso de información satelital, como uno de los métodos para atacar el
  problema mencionado, se ha estado utilizando desde hace algunos años [cita].
  Ésta técnica permite modelar la evolución temporal y geográfica de las
  poblaciones del vector utilizando variables ambientales obtenidas de los
  sensores remotos. Aunque hasta ahora, estos trabajos utilizaban fuertes asunciones
  al utilizar modelos lineales para relacionar las distintas variables [citas], y por más
  que los resultados obtenidos hasta el momento han sido bastante
  favorables, es simple notar que éstos se podrían mejorar sin asumir dichas relaciones.
  Una de la maneras de evitarlo es la utilización de modelos no-lineales de
  Aprendizaje Automático.

\par Desarrollar un Modelo de Aprendizaje Automático puede resultar extremadamente
  complejo y costoso en términos computacionales y de experiencia de quien lo lleve
  a cabo. Pero uno de los objetivos de este trabajo es mostrar la accesibilidad,
  en términos de simpleza y costos, de algunas de estas herramientas, sin dejar
  de lado el desempeño en la tarea concreta. A su vez, también existe el importante problema
  de la escases de datos de campo para utilizarlos en la construcción de los modelos.
  Hasta ahora, era un gran limitante ya que no se tienen datos vitales
  para el desarrollo de este tipo de herramientas. En este trabajo, además, se
  propone una técnica para atenuar dicho problema estableciendo una relación
  entre los distintos puntos geográficos, en función de sus características ambientales.


  ------------ ACA ESTABA ----------------------------------


el Aprendizaje
Automático es un enfoque que se está utilizando

En ese marco, es que se realiza este trabajo: intentando brindar herramientas
que sean de utilidad, como un aporte de información más, a la hora de tomar
decisiones que respecten al control de estas enfermedades. [5], [6]





En la actualidad el dengue es uno de los principales problemas de salud pública en el mundo.
La  Organización  Mundial  de  la  Salud  (OMS)  estima  que  80  millones  de  personas  se  infectan
anualmente,  y  cerca  de  550  mil  enfermos  necesitan  de  hospitalización,  20  mil  mueren  como
consecuencia de dengue, más de 2.500 millones de personas en riesgo de contraer la enfermedad
y más de 100 países tienen transmisión endémica. Se estima que para el año 2085 el cam-
bio climático pondrá a 3.500 millones de personas en riesgo.


En  el  año  2008  se  observó  una  tendencia  ascendente  de  las  formas  graves  de  dengue.  A  fi-
nales  de  2008  en  los  países  americanos  se  han  registrado  854,134  casos,  con  38,627  dengue
DH,SSD,DCC y 584 muertes (tasa de letalidad de 1,5\%). Durante el  primer semestre del presente
año, se han reportado 571,224 casos de dengue, 10,111 casos de dengue hemorrágico y 200
fallecidos con una tasa de letalidad de 1,98\%. La presencia de los 4 serotipos del dengue (DEN
1,2,3,4) circulando en el continente, elevan el riesgo de las formas graves del dengue.


Los grandes cambios demográficos, que han dado por resultado una gran ampliación desorganizada
de las zonas urbanas, junto con el aumento del uso de recipientes no biodegradables
y un método deficitario de recolección de residuos sólidos, incrementan el número de recipien-
tes que acumulan agua, y que actúan como criaderos potenciales del vector, lo cual aumenta el
riesgo de ocurrencia de casos de dengue.


Por  otro  lado,  la  gran  capacidad  adaptativa  del  vector,  el  uso  intensivo  de  insecticidas  con  la
consecuente aparición de resistencia, el cambio climático y la circulación de los cuatro seroti-
pos del virus DEN en las Américas complican día a día la situación



Los fenómenos derivados del calentamiento global conducen a diferentes combinaciones de
cambios  de  temperatura  y  humedad  cuyas  repercusiones  son  heterogéneas  en  la  incidencia
del  dengue  tanto  en  lo  urbano  como  en  lo  rural,  aspectos  que  requieren  mayores  esfuerzos
entre actores sociales nacionales y jurisdiccionales para una mayor gobernabilidad ambiental.
La estacionalidad de la transmisión es un aspecto a considerar teniendo en cuenta que los vec-
tores han desarrollado estrategias para sobrevivir el invierno y en periodos de sequía. Si bien el
clima es un determinante de esa estacionalidad, las condiciones materiales de vida y el  entorno
físico son modificadores importantes del clima a nivel de campo por la variedad de microclimas
que se conforman donde el Aedes aegypti utiliza estrategias para explotar y maximizar las ventajas a su favor


Desde la reintroducción del virus en 1997-98, el dengue avanza sobre la geografía argentina,
presentándose en forma de brotes esporádicos relacionados con la situación epidemiológica
de otros países y restringido a los meses de mayor temperatura.



Hasta  el  año  2008,  cinco  provincias  habían  presentado  casos  de  dengue  autóctonos  con  la
circulación de tres de los cuatro serotipos existentes. Hasta Junio de 2009, la
cantidad de provincias con circulación viral autóctona asciende a 14



Lo mencionado en párrafos anteriores, hace evidente la necesidad de enfocar esfuerzos
en el desarrollo de estrategias contundentes de parte del Estado y las organizaciones
pertinentes dirigidas a evitar, limitar y controlar las poblaciones de \textit{Aedes aegyti}.
En ese marco, es que se realiza este trabajo: intentando brindar herramientas
que sean de utilidad, como un aporte de información más, a la hora de tomar
decisiones que respecten al control de estas enfermedades.

Lo anterior hace prioritario el desarrollo de  estrategias operacionales de campo, estratificadas,
participativas,  complementarias  y  sostenibles  según  grados  de  riesgo  de  transmisión  (condi-
ciones  socio-ambientales,  movilidad  poblacional,  disponibilidad  de  agua,  períodos  del  año,
niveles de infestación, notificaciones, entre otros), para incrementar la eficacia de las interven-
ciones dirigidas a evitar, limitar o controlar las poblaciones de
Aedes aegyti


En  este  escenario  nacional  e  internacional,  se  hace  necesaria  la  participación  multisectorial,
estableciendo unidades  de  apoyo  técnico  y  científico  que  trabajen  bajo
las  premisas de:  Cogestión, Participación Social, Solidaridad y Equidad, conjuntamente con las
acciones de promoción de la salud, prevención y control de estas patologías.


------------------------------------------------------

En Argentina en las semanas del 2018 ya hay casos confirmados de Dengue en Chaco,
y durante el 2017 y en base a las notificaciones al
\textbf{Sistema Nacional de Vigilancia de Salud} del Ministerio de Salud de la Nación
recibidas hasta el 30 de diciembre se registraron en el primer semestre del año brotes de
dengue serotipo DEN-1 con 646 casos confirmados en 5 provincias
(Buenos Aires, Chaco, Corrientes, Formosa y Santa Fe) y 253 casos de enfermedad
por virus del Zika en 3 provincias (Chaco, Formosa y Salta).
Desde la emergencia del virus del Zika en Argentina en el 2016 (Tucumán) y hasta
la SE 47 de 2017 se registraron además un total de 7 casos confirmados de
síndrome congénito asociados a virus del Zika en mujeres embarazadas (microcefalia en recién nacidos).

\par En el caso de la Fiebre Amarilla, para la prevencion, existe una vacuna de
virus atenuado que se considera eficaz, segura y se la utiliza hace más
de 60 años para la inmunización activa de niños y adultos contra la infección
por dicho virus. No es así el caso del Dengue, Chikungunya y Zika, para las
cuales no existe una vacuna para la prevención.


Los mosquitos son los vectores más importantes de enfermedades humanas en el
mundo y, en particular, el Aedes aegypti es el principal vector de dengue,
chikungunya, zika y fiebre amarilla urbana [1]. Como no hay vacunas para la
mayoría de estos virus, y existe la posibilidad de introducción de otros [2],
el control de vectores es la principal herramienta para mitigar la
propagación de enfermedades. La cantidad de vectores, el virus circulante y la
susceptibilidad humana son factores que influyen en la transmisión de estas
enfermedades virales, estos factores poseen una asociación directa o indirecta
con variables climáticas y ambientales tales como la temperatura, la lluvia,
la vegetación, etc. En particular, en Argentina, \textit{Aedes aegypti} es el
mosquito más relevante desde el punto de vista epidemiológico.
Este mosquito se caracteriza por su presencia en el medio urbano, su preferencia
de cría en contenedores artificiales [3] y la resistencia de sus huevos a la
desecación. Para conocer parámetros sobre las distribuciones espaciales y temporales
de \textit{Aedes aegypti}, se utilizan a menudo las Ovitrampas [4].
Estos datos permiten el conocimiento de la actividad del vector, lo cual es importante para determinar cuándo y
dónde aplicar acciones de control más efectivas [5], [6].




En las Américas, entre 1985 y 2012, el 95\% de los casos se concentraron en
4 países: Perú (54\% de los casos), Bolivia (18\%), Brasil (16\%) y Colombia (7\%).
Los otros países en las Américas que presentan condiciones para la transmisión
de fiebre amarilla son Argentina, Ecuador, Guyana, Guyana Francesa, Panamá,
Paraguay, Suriname, Trinidad y Tobago y Venezuela.
Desde 2000 a 2013, más de 1.100 casos confirmados por laboratorio fueron
reportados en las Américas. Brasil y Perú fueron los países que más casos
reportaron. En áreas urbanas, el Aedes aegypti es el mosquito vector de la
fiebre amarilla. La fiebre amarilla selvática es transmitida por los mosquitos
Haemagogus y Sabethes.

La fiebre amarilla se puede prevenir con la vacuna de virus atenuado de fiebre
amarilla, cepa 17D, que se considera eficaz y segura, y se la utiliza hace más
de 60 años para la inmunización activa de niños y adultos contra la infección
por el virus de la fiebre amarilla. Confiere inmunidad duradera, quizá para toda
la vida. No es el caso del Dengue, Zika y Chikungunya, para las cuales no existen
vacunas para el mismo objetivo.


"[...] El escenario epidémico de Arbovirus como Dengue, Zika y Chikungunya es una
realidad epidemiológica que vino para quedarse en Argentina que implica repensar
y diseñar nuevos sistemas de alerta temprana, vigilancia epidemiológica y
respuesta rápida desde lo local que se integren en un espacio intersectorial de
coordinación, planificación e intervención pública nacional entre municipios,
organizaciones de la sociedad civil, universidades y centros de estudios, y
otros actores sociales claves con una necesaria rectoría del
Ministerio de Salud de la Nación y el Consejo Federal de Salud.
Argentina entro a un impasse en este nuevo periodo inter-epidémicos y realmente
sigue sin tener acciones integradas e integrales para el abordaje de las
Enfermedades Transmitidas por Mosquitos (ETM) en el país y se visualizan
cambios en las estrategias y políticas", alerta Medicos del Mundo [citar].



En Argentina en las semanas del 2018 ya hay casos confirmados de Dengue en Chaco,
y durante el 2017 y en base a las notificaciones al
\textbf{Sistema Nacional de Vigilancia de Salud} del MSN recibidas hasta el
30 de diciembre (SE 52) se registraron en el primer semestre del año brotes de
dengue serotipo DEN-1 con 646 casos confirmados en 5 provincias
(Buenos Aires, Chaco, Corrientes, Formosa y Santa Fe) y 253 casos de enfermedad
por virus del Zika en 3 provincias (Chaco, Formosa y Salta).
Desde la emergencia del virus del Zika en Argentina en el 2016 (Tucumán) y hasta
la SE 47 de 2017 se registraron además un total de 7 casos confirmados de
síndrome congénito asociados a virus del Zika en mujeres embarazadas (microcefalia en recién nacidos).



Médicos del Mundo, viene advirtiendo que las acciones implementadas aleatoriamente
de control vectorial y control químico tradicional no son efectivas ni eficaces
para actual contexto epidémico en la región y el país.




"Aunque las últimas epidemias del 2009 y 2016 de Dengue en Argentina fueron del
serotipo DEN1, la circulación viral de los otros serotipos en la región de
Cono Sur (Brasil, Paraguay y Bolivia) tanto DEN4, DEN2 y DEN3, hace que los
periodos epidémicos de DEN se puedan modificar. Por otro lado, el escenario de
Zika Virus es una realidad por su circulación en América Latina y Caribe con
cuadros clínicos inéspecíficos pero con eventos asociados como el Síndrome de
Guillaen Barré y microcefalia que implican problemas epidemiológicos
poblacionales de incidencia como lo demostraron en Brasil, Colombia, Venezuela,
República Dominicana, entre otros 47 países de la región donde se confirmaron
casos de transmisión activa vectorial de Zika.
Si sumamos ahora el brote epidémico de Fiebre Amarilla en Brasil con la
posibilidad de reintroducir casos en el Cono Sur ya que las tasas de
inmunizaciones para fiebre amarrilla existen brechas en varias ciudades de nuestro país",
detalló Gonzalo Basile, Presidente Honor y Director General para
América Latina y Caribe de Médicos del Mundo, e investigador de institutos de
investigación en salud pública del Caribe y coordinación regional del Programa
de Salud Internacional de CLACSO y de FLACSO República Dominicana.





"Desde Médicos del Mundo venimos advirtiendo en lo regional que para el abordaje
de las Arbovirosis hay que repensar nuevos marcos conceptuales y metodológicos
sobre la Salud y la Epidemiología en las Ciudades por el tipo de urbanizaciones
caóticas, inequitativas y con déficit en agua, gestión integral de residuos y
saneamiento ambiental estructurales de enfermedades transmitidas por mosquitos.
Y en segundo lugar, los determinantes vinculados a los patrones climáticos
extremos que generan comportamientos nuevos de los vectores y transmisión de
enfermedades especialmente como Zika, Dengue, Malaria, entre otras.
No es posible continuar haciendo lo mismo que hace 40 años hacemos con Dengue en
la región y esperar resultados diferentes. La transferencia individual de
información y la "responsabilización" individual en los estilos de vida y
comportamientos particulares de las personas está demostrado no son el camino
para enfrentar escenarios epidemiológicos que seguirán complejizándose en el país
y en América Latina y Caribe", concluyó Gonzalo Basile.





Específicamente,

Specifically, in a interinstitutional framework between the Argentinean
National Space Agency (CONAE) and the Health Minister of Argentine,
there have been initiatives to model the temporal evolution of mosquito pop-
ulations using environmental variables obtained from remote sensors. These
works used series of a few years and are based on a small number of satellite
variables [36, 37]. In an effort to improve this, (author?) [38] constructed
models based on a large number of variables from various sensors for four
years. All these works assumed multivariate linear models.
This work represent an improvement of that scenario. We compare Sup-
port Vector Machines, Artificial Neural Networks, K-nearest neighbors and
Decision Tree Regressor in addition to two linear approaches. With this,
we obtain an operational methodology which contributes to the Argentinean
Dengue risk system currently in operation [31, 39].
We explore, in contrast to previous ones, the ability of modeling and
predicting oviposition with out of the shelf ML algorithms, i.e., with min-
imum parameter tuning, as provided by FLOSS – Free/Libre Open Source
Software. This promotes the assimilation of these techniques for the whole
community that deals with similar problems.



--------------------------------







\end{document}
