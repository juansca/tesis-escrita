\documentclass[12pt,spanish,fleqn,openany,letterpaper,pagesize]{scrbook}

\usepackage[utf8]{inputenc}
\usepackage[spanish]{babel}
\usepackage{fancyhdr}
\usepackage{epsfig}
\usepackage{epic}
\usepackage{eepic}
\usepackage{amsmath}
\usepackage{threeparttable}
\usepackage{amscd}
\usepackage{here}
\usepackage{graphicx}
\usepackage{lscape}
\usepackage{tabularx}
\usepackage{subfigure}
\usepackage{longtable}


\usepackage{rotating} %Para rotar texto, objetos y tablas seite. No se ve en DVI solo en PS. Seite 328 Hundebuch
                        %se usa junto con \rotate, \sidewidestable ....


\renewcommand{\theequation}{\thechapter-\arabic{equation}}
\renewcommand{\thefigure}{\textbf{\thechapter-\arabic{figure}}}
\renewcommand{\thetable}{\textbf{\thechapter-\arabic{table}}}


\pagestyle{fancyplain}%\addtolength{\headwidth}{\marginparwidth}
\textheight22.5cm \topmargin0cm \textwidth16.5cm
\oddsidemargin0.5cm \evensidemargin-0.5cm%
\renewcommand{\chaptermark}[1]{\markboth{\thechapter\; #1}{}}
\renewcommand{\sectionmark}[1]{\markright{\thesection\; #1}}
\lhead[\fancyplain{}{\thepage}]{\fancyplain{}{\rightmark}}
\rhead[\fancyplain{}{\leftmark}]{\fancyplain{}{\thepage}}
\fancyfoot{}
\thispagestyle{fancy}%


\addtolength{\headwidth}{0cm}
\unitlength1mm %Define la unidad LE para Figuras
\mathindent0cm %Define la distancia de las formulas al texto,  fleqn las descentra
\marginparwidth0cm
\parindent0cm %Define la distancia de la primera linea de un parrafo a la margen

%Para tablas,  redefine el backschlash en tablas donde se define la posici\'{o}n del texto en las
%casillas (con \centering \raggedright o \raggedleft)
\newcommand{\PreserveBackslash}[1]{\let\temp=\\#1\let\\=\temp}
\let\PBS=\PreserveBackslash

%Espacio entre lineas
\renewcommand{\baselinestretch}{1.1}

%Neuer Befehl f\"{u}r die Tabelle Eigenschaften der Aktivkohlen
\newcommand{\arr}[1]{\raisebox{1.5ex}[0cm][0cm]{#1}}

%Neue Kommandos
\usepackage{Befehle}


%Trennungsliste
\hyphenation {Reaktor-ab-me-ssun-gen Gas-zu-sa-mmen-set-zung
Raum-gesch-win-dig-keit Durch-fluss Stick-stoff-gemisch
Ad-sorp-tions-tem-pe-ra-tur Klein-schmidt
Kohlen-stoff-Mole-kular-siebe Py-rolysat-aus-beu-te
Trans-port-vor-gan-ge}


\begin{document}

\chapter{Motivación}

AAAGREGAR LA IMAGEN DEL PAIS DE LA TESIS DEL GONZA!!!!!
PONER LA IMAGEN DE DENGUE EN 2016 del paper Analytical report of the 2016 dengue outbreak in Córdoba city, Argentina
\justifying
\par El mosquito es uno de los vectores de enfermedades humanas más importantes
  en el mundo. En particular, el \textit{Aedes aegypti} es el principal vector
  de Dengue, Chikungunya, Zika y Fiebre Amarilla urbana [1].
  Según datos de la Organización Mundial de la Salud (OMS), alrededor de 80 millones de
  personas se infectan de Dengue anualmente, cerca de 550 mil enfermos requieren hospitalización y
  unos 20 mil mueren. Además, calculan que más de 2.500 millones de personas corren
  riesgo de contraer la enfermedad y más de 100 países tienen transmisión endémica.
  [Directrices para la prevención y control de Aedes aegypti, está el pdf]
  Algo que cabe aclarar es que en el caso de la Fiebre Amarilla, para la
  prevencion, existe una vacuna de virus atenuado que se considera eficaz, segura
  y se la utiliza hace más de 60 años para la inmunización activa de niños y
  adultos. No es así el caso del Dengue,
  Chikungunya y Zika, para las cuales no existe tal herramienta de previsión.

\par Si tenemos en cuenta las 4 enfermedades mencionadas en el parrafo anterior,
  en las Américas, entre 1985 y 2012, el 95\% de los casos se concentraron en
  4 países: Perú (54\% de los casos), Bolivia (18\%), Brasil (16\%) y Colombia (7\%).
  Otras naciones de la región que presentan condiciones muy
  favorables para la transimisión son Argentina, Ecuador, Panamá y Venezuela.
  Desde 2000 a 2013, más de 1.100 casos confirmados por laboratorio fueron
  reportados.
  [http://www.mdm.org.ar/prensa/articulo/221/Mdicos-del-Mundo-alerta-sobre-riesgos-de-fiebre-amarilla-en-Brasil-y-escenarios-de-Dengue-Zika-en-Argentina\#.W1yUa6zV-AU.link]

\par En el caso de Argentina, ya para las primeras semanas del 2018, hubo casos confirmados
  de Dengue en Chaco, y durante el 2017, en base a las notificaciones al
  \textbf{Sistema Nacional de Vigilancia de Salud} del Ministerio de Salud de la Nación
  recibidas hasta el 30 de diciembre, se registraron, en el primer semestre del año, brotes de
  Dengue serotipo DEN-1 con 646 casos confirmados en 5 provincias
  (Buenos Aires, Chaco, Corrientes, Formosa y Santa Fe) y 253 casos de enfermedad
  por virus del Zika en 3 provincias (Chaco, Formosa y Salta).
  Desde la emergencia del virus del Zika en nuestro país en el 2016 (Tucumán), y hasta
  la [SE 47 REVISAR] de 2017 se registraron además un total de 7 casos confirmados de
  síndrome congénito asociados a virus del Zika en mujeres embarazadas
  (microcefalia en recién nacidos).

\par El Dr. Gonzalo Basile \footnote{Presidente Honor y Director General para
                 América Latina y Caribe de Médicos del Mundo, e investigador de institutos de
                 investigación en salud pública del Caribe y coordinación regional del Programa
                 de Salud Internacional de CLACSO y de FLACSO República Dominicana}
     se refiere al incremento del riesgo de crecimiento
     en la cantidad de casos positivos en nuestro país, teniendo en cuenta
     el contexto epidemiológico en la región:
\begin{framed}

  Aunque las últimas epidemias del 2009 y 2016 de Dengue en Argentina fueron del
  serotipo DEN1, la circulación viral de los otros serotipos en la región de
  Cono Sur (Brasil, Paraguay y Bolivia) tanto DEN4, DEN2 y DEN3, hace que los
  periodos epidémicos de DEN se puedan modificar. Por otro lado, el escenario de
  Zika Virus es una realidad por su circulación en América Latina y Caribe con
  cuadros clínicos inéspecíficos pero con eventos asociados como el Síndrome de
  Guillaen Barré y microcefalia que implican problemas epidemiológicos
  poblacionales de incidencia como lo demostraron en Brasil, Colombia, Venezuela,
  República Dominicana, entre otros 47 países de la región donde se confirmaron
  casos de transmisión activa vectorial de Zika.
  Si sumamos ahora el brote epidémico de Fiebre Amarilla en Brasil con la
  posibilidad de reintroducir casos en el Cono Sur ya que las tasas de
  inmunizaciones para fiebre amarrilla existen brechas en varias ciudades de nuestro país \\

 \centering 24/01/2018
\end{framed}



\par Sumado a lo comentado, por su parte, el \textit{Aedes aegypti} se
  caracteriza por su presencia en el medio urbano, su preferencia
  de cría en contenedores artificiales [3] y la resistencia de sus huevos a la
  desecación. En nuestro país, además, los vertiginosos cambios demográficos, han dado
  por resultado una gran ampliación desorganizada de las zonas urbanas. Ésto, junto
  con el aumento del uso de recipientes no biodegradables y un método deficiente
  de recolección de residuos sólidos, incrementan el número de depósitos que
  acumulan agua, que actúan como potenciales criaderos del mosquito, lo cual aumenta el
  riesgo de ocurrencia de casos de las enfermedades mencionadas.
  Dado que la cantidad de vectores, el virus circulante y la susceptibilidad
  humana dependen directa o indirectamente de variables climáticas y ambientales [buscar cita]
  tales como la temperatura, la lluvia, la vegetación, entre otras, es razonable
  suponer que el cambio climático es, también, un factor de riesgo para el desarrollo
  de las enfermedades en cuestión. Por otra parte, se le suma la capacidad adaptativa del
  \textit{Aedes aegypti} y la aparición de resistencia del mismo debido al uso intensivo de
  insecticidas.


\par Por otro lado, como mencionamos anteriormente, dado que no hay vacunas para la
  mayoría de estos virus, y existe la posibilidad de introducción de otros [2],
  el control de vectores es la principal herramienta para mitigar la
  propagación de enfermedades.


\par Es claro que el escenario epidémico planteado es una realidad en Argentina
  que hay que atacar. Ésto deriva en la necesidad de enfocar esfuerzos en el
  desarrollo de estrategias contundentes dirigidas a evitar, limitar y controlar
  las poblaciones de \textit{Aedes aegyti}, lo que implica repensar y diseñar
  nuevos sistemas de alerta temprana, vigilancia epidemiológica y respuesta
  rápida desde lo local, integrando un espacio interinstitutional e
  intersectorial de coordinación, planificación e intervención pública. Ésto debe
  llevarse a cabo entre el Estado Nacional, municipios, universidades y centros
  de estudio, organizaciones civiles, entre otros actores sociales de gran importancia.
  En ese contexto, la introducción de herramientas científico/tecnológicas orientadas
  a contribuir en esos aspectos resulta fundamental.

\par El uso de información satelital, como uno de los métodos para atacar el
  problema mencionado, se ha estado utilizando desde hace algunos años [cita].
  Ésta técnica permite modelar la evolución temporal y geográfica de las
  poblaciones del vector utilizando variables ambientales obtenidas de los
  sensores remotos. Aunque hasta ahora, estos trabajos utilizaban fuertes asunciones
  al utilizar modelos lineales para relacionar las distintas variables [citas], y por más
  que los resultados obtenidos hasta el momento han sido bastante
  favorables, es simple notar que éstos se podrían mejorar sin asumir dichas relaciones.
  Una de la maneras de evitarlo es la utilización de modelos no-lineales de
  Aprendizaje Automático.

\par Desarrollar un Modelo de Aprendizaje Automático puede resultar extremadamente
  complejo y costoso en términos computacionales y de experiencia de quien lo lleve
  a cabo. Pero uno de los objetivos de este trabajo es mostrar la accesibilidad,
  en términos de simpleza y costos, de algunas de estas herramientas, sin dejar
  de lado el desempeño en la tarea concreta. A su vez, también existe el importante problema
  de la escases de datos de campo para utilizarlos en la construcción de los modelos.
  Hasta ahora, era un gran limitante ya que no se tienen datos vitales
  para el desarrollo de este tipo de herramientas. En este trabajo, además, se
  propone una técnica para atenuar dicho problema estableciendo una relación
  entre los distintos puntos geográficos, en función de sus características ambientales.


\end{document}
