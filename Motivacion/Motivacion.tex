\documentclass[12pt,spanish,fleqn,openany,letterpaper,pagesize]{scrbook}

\usepackage[utf8]{inputenc}
\usepackage[spanish]{babel}
\usepackage{fancyhdr}
\usepackage{epsfig}
\usepackage{epic}
\usepackage{eepic}
\usepackage{amsmath}
\usepackage{amssymb}
\usepackage{siunitx}
\usepackage{booktabs}
\usepackage{lineno}
\usepackage{threeparttable}
\usepackage{amscd}
\usepackage{here}
\usepackage{graphicx}
\usepackage{lscape}
\usepackage{tabularx}
\usepackage{url}
\usepackage{subfigure}
\usepackage{longtable}
\usepackage{framed}
\usepackage{ragged2e} \justifying
\usepackage{rotating} %Para rotar texto, objetos y tablas seite. No se ve en DVI solo en PS. Seite 328 Hundebuch
                        %se usa junto con \rotate, \sidewidestable ....
\usepackage{tikz}
\usepackage{verbatim}


\usepackage{bbm}

\usepackage{listings}
\usepackage{color}

\definecolor{dkgreen}{rgb}{0,0.6,0}
\definecolor{gray}{rgb}{0.5,0.5,0.5}
\definecolor{mauve}{rgb}{0.58,0,0.82}

\lstset{frame=tb,
  language=Bash,
  aboveskip=3mm,
  belowskip=3mm,
  showstringspaces=false,
  columns=flexible,
  basicstyle={\small\ttfamily},
  numbers=none,
  numberstyle=\tiny\color{gray},
  keywordstyle=\color{blue},
  commentstyle=\color{dkgreen},
  stringstyle=\color{mauve},
  breaklines=true,
  breakatwhitespace=true,
  tabsize=3
}

\renewcommand{\theequation}{\thechapter-\arabic{equation}}
\renewcommand{\thefigure}{\textbf{\thechapter-\arabic{figure}}}
\renewcommand{\thetable}{\textbf{\thechapter-\arabic{table}}}
\newcommand{\norm}[1]{\left\lVert#1\right\rVert}

\pagestyle{fancyplain}%\addtolength{\headwidth}{\marginparwidth}
\textheight22.5cm \topmargin0cm \textwidth16.5cm
\oddsidemargin0.5cm \evensidemargin-0.5cm%
\renewcommand{\chaptermark}[1]{\markboth{\thechapter\; #1}{}}
\renewcommand{\sectionmark}[1]{\markright{\thesection\; #1}}
\lhead[\fancyplain{}{\thepage}]{\fancyplain{}{\rightmark}}
\rhead[\fancyplain{}{\leftmark}]{\fancyplain{}{\thepage}}
\fancyfoot{}
\thispagestyle{fancy}%


\addtolength{\headwidth}{0cm}
\unitlength1mm %Define la unidad LE para Figuras
\mathindent0cm %Define la distancia de las formulas al texto,  fleqn las descentra
\marginparwidth0cm
\parindent0cm %Define la distancia de la primera linea de un parrafo a la margen

%Para tablas,  redefine el backschlash en tablas donde se define la posici\'{o}n del texto en las
%casillas (con \centering \raggedright o \raggedleft)
\newcommand{\PreserveBackslash}[1]{\let\temp=\\#1\let\\=\temp}
\let\PBS=\PreserveBackslash

%Espacio entre lineas
\renewcommand{\baselinestretch}{1.1}

%Neuer Befehl f\"{u}r die Tabelle Eigenschaften der Aktivkohlen
\newcommand{\arr}[1]{\raisebox{1.5ex}[0cm][0cm]{#1}}

%Neue Kommandos
\usepackage{Befehle}


%Trennungsliste
\hyphenation {Reaktor-ab-me-ssun-gen Gas-zu-sa-mmen-set-zung
Raum-gesch-win-dig-keit Durch-fluss Stick-stoff-gemisch
Ad-sorp-tions-tem-pe-ra-tur Klein-schmidt
Kohlen-stoff-Mole-kular-siebe Py-rolysat-aus-beu-te
Trans-port-vor-gan-ge}


\begin{document}

\chapter{Motivación}

\par El mosquito es uno de los vectores de enfermedades humanas más importantes
en el mundo. En particular, el \textit{Aedes aegypti} es el principal vector
de Dengue, Chikungunya, Zika y Fiebre Amarilla urbana [1].

\par En América Latina, entre 1985 y 2012, el 95\% de los casos se concentraron en
4 países: Perú (54\% de los casos), Bolivia (18\%), Brasil (16\%) y Colombia (7\%).
Los otros paises que presentan condiciones muy favorables para la transimisión
de fiebre amarilla son Argentina, Ecuador, Panamá y Venezuela
Desde 2000 a 2013, más de 1.100 casos confirmados por laboratorio fueron
reportados. Como vimos, Brasil y Perú fueron los paises que más casos reportaron.


\par En Argentina, ya para las primeras semanas del 2018, hubo casos confirmados
de Dengue en Chaco, y durante el 2017, en base a las notificaciones al
\textbf{Sistema Nacional de Vigilancia de Salud} del Ministerio de Salud de la Nación
recibidas hasta el 30 de diciembre se registraron en el primer semestre del año brotes de
Dengue serotipo DEN-1 con 646 casos confirmados en 5 provincias
(Buenos Aires, Chaco, Corrientes, Formosa y Santa Fe) y 253 casos de enfermedad
por virus del Zika en 3 provincias (Chaco, Formosa y Salta).
Desde la emergencia del virus del Zika en Argentina en el 2016 (Tucumán) y hasta
la SE 47 de 2017 se registraron además un total de 7 casos confirmados de
síndrome congénito asociados a virus del Zika en mujeres embarazadas (microcefalia en recién nacidos).
------------ ACA ESTABA ----------------------------------

En Argentina en las semanas del 2018 ya hay casos confirmados de Dengue en Chaco,
y durante el 2017 y en base a las notificaciones al
\textbf{Sistema Nacional de Vigilancia de Salud} del Ministerio de Salud de la Nación
recibidas hasta el 30 de diciembre se registraron en el primer semestre del año brotes de
dengue serotipo DEN-1 con 646 casos confirmados en 5 provincias
(Buenos Aires, Chaco, Corrientes, Formosa y Santa Fe) y 253 casos de enfermedad
por virus del Zika en 3 provincias (Chaco, Formosa y Salta).
Desde la emergencia del virus del Zika en Argentina en el 2016 (Tucumán) y hasta
la SE 47 de 2017 se registraron además un total de 7 casos confirmados de
síndrome congénito asociados a virus del Zika en mujeres embarazadas (microcefalia en recién nacidos).

\par En el caso de la Fiebre Amarilla, para la prevencion, existe una vacuna de
virus atenuado que se considera eficaz, segura y se la utiliza hace más
de 60 años para la inmunización activa de niños y adultos contra la infección
por dicho virus. No es así el caso del Dengue, Chikungunya y Zika, para las
cuales no existe una vacuna para la prevención.


Los mosquitos son los vectores más importantes de enfermedades humanas en el
mundo y, en particular, el Aedes aegypti es el principal vector de dengue,
chikungunya, zika y fiebre amarilla urbana [1]. Como no hay vacunas para la
mayoría de estos virus, y existe la posibilidad de introducción de otros [2],
el control de vectores es la principal herramienta para mitigar la
propagación de enfermedades. La cantidad de vectores, el virus circulante y la
susceptibilidad humana son factores que influyen en la transmisión de estas
enfermedades virales, estos factores poseen una asociación directa o indirecta
con variables climáticas y ambientales tales como la temperatura, la lluvia,
la vegetación, etc. En particular, en Argentina, \textit{Aedes aegypti} es el
mosquito más relevante desde el punto de vista epidemiológico.
Este mosquito se caracteriza por su presencia en el medio urbano, su preferencia
de cría en contenedores artificiales [3] y la resistencia de sus huevos a la
desecación. Para conocer parámetros sobre las distribuciones espaciales y temporales
de \textit{Aedes aegypti}, se utilizan a menudo las Ovitrampas [4].
Estos datos permiten el conocimiento de la actividad del vector, lo cual es importante para determinar cuándo y
dónde aplicar acciones de control más efectivas [5], [6].




En las Américas, entre 1985 y 2012, el 95\% de los casos se concentraron en
4 países: Perú (54\% de los casos), Bolivia (18\%), Brasil (16\%) y Colombia (7\%).
Los otros países en las Américas que presentan condiciones para la transmisión
de fiebre amarilla son Argentina, Ecuador, Guyana, Guyana Francesa, Panamá,
Paraguay, Suriname, Trinidad y Tobago y Venezuela.
Desde 2000 a 2013, más de 1.100 casos confirmados por laboratorio fueron
reportados en las Américas. Brasil y Perú fueron los países que más casos
reportaron. En áreas urbanas, el Aedes aegypti es el mosquito vector de la
fiebre amarilla. La fiebre amarilla selvática es transmitida por los mosquitos
Haemagogus y Sabethes.

La fiebre amarilla se puede prevenir con la vacuna de virus atenuado de fiebre
amarilla, cepa 17D, que se considera eficaz y segura, y se la utiliza hace más
de 60 años para la inmunización activa de niños y adultos contra la infección
por el virus de la fiebre amarilla. Confiere inmunidad duradera, quizá para toda
la vida. No es el caso del Dengue, Zika y Chikungunya, para las cuales no existen
vacunas para el mismo objetivo.


"[...] El escenario epidémico de Arbovirus como Dengue, Zika y Chikungunya es una
realidad epidemiológica que vino para quedarse en Argentina que implica repensar
y diseñar nuevos sistemas de alerta temprana, vigilancia epidemiológica y
respuesta rápida desde lo local que se integren en un espacio intersectorial de
coordinación, planificación e intervención pública nacional entre municipios,
organizaciones de la sociedad civil, universidades y centros de estudios, y
otros actores sociales claves con una necesaria rectoría del
Ministerio de Salud de la Nación y el Consejo Federal de Salud.
Argentina entro a un impasse en este nuevo periodo inter-epidémicos y realmente
sigue sin tener acciones integradas e integrales para el abordaje de las
Enfermedades Transmitidas por Mosquitos (ETM) en el país y se visualizan
cambios en las estrategias y políticas", alerta Medicos del Mundo [citar].



En Argentina en las semanas del 2018 ya hay casos confirmados de Dengue en Chaco,
y durante el 2017 y en base a las notificaciones al
\textbf{Sistema Nacional de Vigilancia de Salud} del MSN recibidas hasta el
30 de diciembre (SE 52) se registraron en el primer semestre del año brotes de
dengue serotipo DEN-1 con 646 casos confirmados en 5 provincias
(Buenos Aires, Chaco, Corrientes, Formosa y Santa Fe) y 253 casos de enfermedad
por virus del Zika en 3 provincias (Chaco, Formosa y Salta).
Desde la emergencia del virus del Zika en Argentina en el 2016 (Tucumán) y hasta
la SE 47 de 2017 se registraron además un total de 7 casos confirmados de
síndrome congénito asociados a virus del Zika en mujeres embarazadas (microcefalia en recién nacidos).



Médicos del Mundo, viene advirtiendo que las acciones implementadas aleatoriamente
de control vectorial y control químico tradicional no son efectivas ni eficaces
para actual contexto epidémico en la región y el país.




"Aunque las últimas epidemias del 2009 y 2016 de Dengue en Argentina fueron del
serotipo DEN1, la circulación viral de los otros serotipos en la región de
Cono Sur (Brasil, Paraguay y Bolivia) tanto DEN4, DEN2 y DEN3, hace que los
periodos epidémicos de DEN se puedan modificar. Por otro lado, el escenario de
Zika Virus es una realidad por su circulación en América Latina y Caribe con
cuadros clínicos inéspecíficos pero con eventos asociados como el Síndrome de
Guillaen Barré y microcefalia que implican problemas epidemiológicos
poblacionales de incidencia como lo demostraron en Brasil, Colombia, Venezuela,
República Dominicana, entre otros 47 países de la región donde se confirmaron
casos de transmisión activa vectorial de Zika.
Si sumamos ahora el brote epidémico de Fiebre Amarilla en Brasil con la
posibilidad de reintroducir casos en el Cono Sur ya que las tasas de
inmunizaciones para fiebre amarrilla existen brechas en varias ciudades de nuestro país",
detalló Gonzalo Basile, Presidente Honor y Director General para
América Latina y Caribe de Médicos del Mundo, e investigador de institutos de
investigación en salud pública del Caribe y coordinación regional del Programa
de Salud Internacional de CLACSO y de FLACSO República Dominicana.





"Desde Médicos del Mundo venimos advirtiendo en lo regional que para el abordaje
de las Arbovirosis hay que repensar nuevos marcos conceptuales y metodológicos
sobre la Salud y la Epidemiología en las Ciudades por el tipo de urbanizaciones
caóticas, inequitativas y con déficit en agua, gestión integral de residuos y
saneamiento ambiental estructurales de enfermedades transmitidas por mosquitos.
Y en segundo lugar, los determinantes vinculados a los patrones climáticos
extremos que generan comportamientos nuevos de los vectores y transmisión de
enfermedades especialmente como Zika, Dengue, Malaria, entre otras.
No es posible continuar haciendo lo mismo que hace 40 años hacemos con Dengue en
la región y esperar resultados diferentes. La transferencia individual de
información y la "responsabilización" individual en los estilos de vida y
comportamientos particulares de las personas está demostrado no son el camino
para enfrentar escenarios epidemiológicos que seguirán complejizándose en el país
y en América Latina y Caribe", concluyó Gonzalo Basile.





Específicamente,

Specifically, in a interinstitutional framework between the Argentinean
National Space Agency (CONAE) and the Health Minister of Argentine,
there have been initiatives to model the temporal evolution of mosquito pop-
ulations using environmental variables obtained from remote sensors. These
works used series of a few years and are based on a small number of satellite
variables [36, 37]. In an effort to improve this, (author?) [38] constructed
models based on a large number of variables from various sensors for four
years. All these works assumed multivariate linear models.
This work represent an improvement of that scenario. We compare Sup-
port Vector Machines, Artificial Neural Networks, K-nearest neighbors and
Decision Tree Regressor in addition to two linear approaches. With this,
we obtain an operational methodology which contributes to the Argentinean
Dengue risk system currently in operation [31, 39].
We explore, in contrast to previous ones, the ability of modeling and
predicting oviposition with out of the shelf ML algorithms, i.e., with min-
imum parameter tuning, as provided by FLOSS – Free/Libre Open Source
Software. This promotes the assimilation of these techniques for the whole
community that deals with similar problems.



--------------------------------







\end{document}
