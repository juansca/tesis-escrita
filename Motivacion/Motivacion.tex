\chapter{Motivación}

El Aprendizaje Automático (ML) es un enfoque empírico efectivo para
regresiones y/o clasificaciones de sistemas no-lineales que pueden involucrar desde
unos pocos hasta varios cientos de variables. El enfoque de ML requiere
entrenamiento utilizando un conjunto de datos que sea representativo del conjunto
de datos del problema. Se debe seleccionar un conjunto de datos que se utilizará,
luego del entrenamiento del modelo, para validación del mismo.
ML es ideal para aquellos problemas en donde el conocimiento teórico del mismo
es incompleto o insuficiente, pero se cuenta con un gran conjunto de observaciones.
ML ha mostrado ser de utilidad para un gran número de aplicaciones en Geociencias
relacionadas a la tierra, oceanos y atmósfera, y en algoritmos de extracción
de información bio-geofísica.
Algunos de los algoritmos de ML más usados en aplicaciones relativas a
Geociencias y Sensado Remoto (GRS) son las Redes Neuronales Artificiales (ANN),
Support Vector Machines (SVM), Mapas Auto-organizados (SOM), Árboles de Decisión (DT),
Random Forests y Algoritmos Genéticos. Su aplicación en problemas de GRS es
relativamente nuevo y extremádamente prometedora. En particular, ANNs son
usadas para clasifición pero también son usadas para la aplicación en pronósticos
relativos a series de tiempo.
De hecho, una exploración en la base bibliográfica Scopus devuelve más de 4000
publicaciones que incluyen \textit{remote sensing} y \textit{neural network},
311 de ellas en 2016. De ese total, el 45\% corresponde a
\textit{Sciences of the Earth}, el 44\% a \textit{Computer Science} y el 35\% a
\textit{Engineering}; con China, Estados Unidos, Italia e India como los paises
con mayor producción científica en dichas áreas.


Los mosquitos

Mosquitoes are the most important vectors of human diseases. In partic-
ular, Aedes ægypti is the main vector for Chikungunya, Dengue, and Zika
viruses. This is a peridomestic mosquito that is bred preferably in artificial
containers [15], [16]. The incidence of Dengue has increased dramatically
in the last decades, with a rising trend of outbreaks in South America in
recent years, and Chikungunya and Zika are new threats spread by the same
species of mosquito [17] , [18], [19]. The deployment of ovitraps is generally
accepted as a valid method to provide useful data on the spatial and tempo-
ral distribution of Aedes ægypti, allowing a reasonable estimation of vector
261 activity [20].
Landscape Epidemiology [21, 22] promotes the notion that satellite data
from earth observation and geospatial technologies are essential tools [23] to
address vector borne epidemiological problems. Using these ideas, several
interdisciplinary studies were produced in latinoamerica focused in generat-
ing spatial and temporal predictive risk models based on satellite derived
environmental conditions [24, 25, 26, 27]. In particular in Argentine there
are some interesting experiences on this issue, [28, 29, 30] deal with Dengue
epidemics leading to operational tools [31]. At a global scope we can find
interesting contributions [32, 33, 34] with also some operatives experiences
[35]
Specifically, in a interinstitutional framework between the Argentinean
National Space Agency (CONAE) and the Health Minister of Argentine,
there have been initiatives to model the temporal evolution of mosquito pop-
ulations using environmental variables obtained from remote sensors. These
works used series of a few years and are based on a small number of satellite
variables [36, 37]. In an effort to improve this, (author?) [38] constructed
models based on a large number of variables from various sensors for four
years. All these works assumed multivariate linear models.
This work represent an improvement of that scenario. We compare Sup-
port Vector Machines, Artificial Neural Networks, K-nearest neighbors and
Decision Tree Regressor in addition to two linear approaches. With this,
we obtain an operational methodology which contributes to the Argentinean
Dengue risk system currently in operation [31, 39].
We explore, in contrast to previous ones, the ability of modeling and
predicting oviposition with out of the shelf ML algorithms, i.e., with min-
imum parameter tuning, as provided by FLOSS – Free/Libre Open Source
Software. This promotes the assimilation of these techniques for the whole
community that deals with similar problems.



--------------------------------





Los mosquitos son los vectores más importantes de enfermedades humanas en el
mundo y, en particular, el Aedes aegypti es el principal vector de dengue,
chikungunya, zika y fiebre amarilla urbana [1]. Como no hay vacunas para la
mayoría de estos virus, y existe la posibilidad de introducción de otros [2],
el control de vectores es la principal herramienta para mitigar la
propagación de enfermedades. La cantidad de vectores, el virus circulante y la
susceptibilidad humana son factores que influyen en la transmisión de estas
enfermedades virales, estos factores poseen una asociación directa o indirecta
con variables climáticas y ambientales tales como la temperatura, la lluvia,
la vegetación, etc. En particular, en Argentina, \textit{Aedes aegypti} es el
mosquito más relevante desde el punto de vista epidemiológico.
Este mosquito se caracteriza por su presencia en el medio urbano, su preferencia
de cría en contenedores artificiales [3] y la resistencia de sus huevos a la
desecación. Para conocer parámetros sobre las distribuciones espaciales y temporales
de \textit{Aedes aegypti}, se utilizan a menudo las Ovitrampas [4].
Estos datos permiten el conocimiento de la actividad del vector, lo cual es importante para determinar cuándo y
dónde aplicar acciones de control más efectivas [5], [6].

El concepto interdisciplinario de \textit{Epidemiología Panorámica} (EP)
ha permitido generar algunos progresos en la epidemiología del dengue y otras
enfermedades transmitidas por vectores y zoonóticas como el chagas, la
malaria, la leishmaniasis y el hantavirus. Esta herramienta, EP, se centra en
producir mapas de riesgo predictivos espaciales y temporales basados en
características ambientales junto con datos de campo [7-14].
Estos estudios en Latinoamérica, donde se utiliza la tecnología espacial en
problemas epidemiológicos, están inspirados en trabajos pioneros llevados a cabo
en EE. UU. y Europa [15-18].
Con base en los trabajos citados en el párrafo anterior, en 2011, Argentina
comenzó a desarrollar un proyecto operacional
(Sistema de Alerta Temprana de Salud, HEWS), útil tanto para las autoridades de
salud como para los investigadores.

Básicamente, HEWS es un mapeo de riesgo
dinámico del dengue para todas las ciudades del país. En este producto, cada
ciudad es representada por un punto al que se le asigna un valor de riesgo para
cada año, basado en tecnología geoespacial. El trabajo fue realizado en un
contexto interdisciplinario e interinstitucional.

La plataforma se desarrolló utilizando software de código abierto (OSS) para
generar un servicio gratuito. En este sistema [13], el riesgo
se evalúa en cuatro componentes que son: el entomológico,
el viral, el componente relacionado con las actividades de
control y finalmente el ambiental. Mientras que los tres
primeros componentes se generan con el aporte de
información de los agentes de salud que trabajan en cada
ciudad, el cuarto se evalúa a partir de datos satelitales.
Específicamente el componente ambiental, en la versión
inicial del sistema, se evalúa con una probabilidad
estacionaria de presencia de vectores (igual para todo el
tiempo) más un componente relacionado con el número de
ciclos virales, que son una función de la temperatura, y así
es diferente para cada ciudad y para cada año. El mapa de
probabilidad de presencia de especie (modelo de nicho) es
claramente una gran simplificación y se puede mejorar en
base a datos satelitales continuos del medio ambiente.
Variables como precipitación y temperatura, han
demostrado, con una variabilidad local, influenciar el
desarrollo de mosquitos, su supervivencia y actividad de
oviposición y por ende la abundancia de vectores. En
particular trabajos previos como los de Estallo [8], [19]
utilizan información satelital para estimar la evolución
temporal de la abundancia de vectores. En 2017 German y
colaboradores [20] desarrollan una metodología completa
para generar modelos de manera automática y basada en
información de libre acceso. En particular German [20]
utiliza productos del sensor (MODIS) a bordo del satélite
Terra y Aqua, pues es uno de los más adecuados para esta
aplicación particular, debido a su resolución temporal,
espectral y espacial. MODIS proporciona un conjunto de
productos pre-procesados y de libre acceso [21].
Específicamente, los productos de vegetación (índice de
vegetación de diferencia normalizada) y temperatura
(temperatura de la superficie terrestre) derivados de MODIS
son ejemplos de variables de percepción remota utilizadas
en aplicaciones de epidemiologia [13], [22] incorporadas en
[20]. Otra variable ambiental obtenida de satélite que es
relevante e incorporada, es el Índice de Agua de Diferencia
Normalizada (NDWI) que evalúa de alguna forma el
contenido de agua de la cubierta. Adicionalmente el trabajo
de German incorpora una estimación de la precipitación
desde el espacio a partir de las misiones (TRMM) y (GPM)
[23].
Estos modelos temporales antes citados [8], [19], [20] se
basan en la generación de relaciones empíricas entre datos
satelitales y los datos de campo (del vector). Esto significa
que sólo pueden construirse modelos en lugares donde esté
disponible la información de campo. En este marco, y con
el objetivo final de mejorar la aplicación operativa
presentada por Porcasi y colaboradores en 2012 [13], nos
planteamos en este trabajo el objetivo específico de generar
una metodología para espacializar los modelos temporales
generados siguiendo la metodología de German 2018 [20],
basados en el concepto de Distancia Ambiental
Normalizada (NED).
