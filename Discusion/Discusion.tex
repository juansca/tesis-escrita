\documentclass[12pt,spanish,fleqn,openany,letterpaper,pagesize]{scrbook}

\usepackage[utf8]{inputenc}
\usepackage[spanish]{babel}
\usepackage{fancyhdr}
\usepackage{epsfig}
\usepackage{epic}
\usepackage{eepic}
\usepackage{amsmath}
\usepackage{amssymb}
\usepackage{siunitx}
\usepackage{booktabs}
\usepackage{lineno}
\usepackage{threeparttable}
\usepackage{amscd}
\usepackage{here}
\usepackage{graphicx}
\usepackage{lscape}
\usepackage{tabularx}
\usepackage{url}
\usepackage{subfigure}
\usepackage{longtable}
\usepackage{framed}
\usepackage{ragged2e} \justifying
\usepackage{rotating} %Para rotar texto, objetos y tablas seite. No se ve en DVI solo en PS. Seite 328 Hundebuch
                        %se usa junto con \rotate, \sidewidestable ....
\usepackage{tikz}
\usepackage{verbatim}


\usepackage{bbm}

\usepackage{listings}
\usepackage{color}

\definecolor{dkgreen}{rgb}{0,0.6,0}
\definecolor{gray}{rgb}{0.5,0.5,0.5}
\definecolor{mauve}{rgb}{0.58,0,0.82}

\lstset{frame=tb,
  language=Bash,
  aboveskip=3mm,
  belowskip=3mm,
  showstringspaces=false,
  columns=flexible,
  basicstyle={\small\ttfamily},
  numbers=none,
  numberstyle=\tiny\color{gray},
  keywordstyle=\color{blue},
  commentstyle=\color{dkgreen},
  stringstyle=\color{mauve},
  breaklines=true,
  breakatwhitespace=true,
  tabsize=3
}

\renewcommand{\theequation}{\thechapter-\arabic{equation}}
\renewcommand{\thefigure}{\textbf{\thechapter-\arabic{figure}}}
\renewcommand{\thetable}{\textbf{\thechapter-\arabic{table}}}
\newcommand{\norm}[1]{\left\lVert#1\right\rVert}

\pagestyle{fancyplain}%\addtolength{\headwidth}{\marginparwidth}
\textheight22.5cm \topmargin0cm \textwidth16.5cm
\oddsidemargin0.5cm \evensidemargin-0.5cm%
\renewcommand{\chaptermark}[1]{\markboth{\thechapter\; #1}{}}
\renewcommand{\sectionmark}[1]{\markright{\thesection\; #1}}
\lhead[\fancyplain{}{\thepage}]{\fancyplain{}{\rightmark}}
\rhead[\fancyplain{}{\leftmark}]{\fancyplain{}{\thepage}}
\fancyfoot{}
\thispagestyle{fancy}%


\addtolength{\headwidth}{0cm}
\unitlength1mm %Define la unidad LE para Figuras
\mathindent0cm %Define la distancia de las formulas al texto,  fleqn las descentra
\marginparwidth0cm
\parindent0cm %Define la distancia de la primera linea de un parrafo a la margen

%Para tablas,  redefine el backschlash en tablas donde se define la posici\'{o}n del texto en las
%casillas (con \centering \raggedright o \raggedleft)
\newcommand{\PreserveBackslash}[1]{\let\temp=\\#1\let\\=\temp}
\let\PBS=\PreserveBackslash

%Espacio entre lineas
\renewcommand{\baselinestretch}{1.1}

%Neuer Befehl f\"{u}r die Tabelle Eigenschaften der Aktivkohlen
\newcommand{\arr}[1]{\raisebox{1.5ex}[0cm][0cm]{#1}}

%Neue Kommandos
\usepackage{Befehle}


%Trennungsliste
\hyphenation {Reaktor-ab-me-ssun-gen Gas-zu-sa-mmen-set-zung
Raum-gesch-win-dig-keit Durch-fluss Stick-stoff-gemisch
Ad-sorp-tions-tem-pe-ra-tur Klein-schmidt
Kohlen-stoff-Mole-kular-siebe Py-rolysat-aus-beu-te
Trans-port-vor-gan-ge}


\begin{document}

\justifying

\chapter{Discusión}

  \par Dengue, Chikungunya y Zika son enfermedades virales para las cuales no
    existe, al día de hoy, vacunas de prevención. Por lo tanto, el control
    más efectivo proviene de prevenir la propagación del mosquito
    \textit{Aedes Aegipty} (\textit{Linneaus}) y, por tanto, saber sobre la
    dinámica de su población es de suma importancia. Éste trabajo presenta un
    \textit{framework} para el pronóstico de la oviposición utilizando
    únicamente variables ambientales extraídas de información satelital y
    herramientas de Aprendizaje Automático de libre acceso. Estas herramientas
    son una mejora al sistema operacional de riesgo de
    Argentina \cite{porcasi_operative}.

  \par Utilizamos variables ambientales derivadas de información satelital
    (temperatura, humedad y precipitación) operacionalmente disponibles para
    construir modelos temporales capaces de predecir la actividad de oviposición
    fuera de las casas. En ese sentido, nuestra perspectiva, completamente operativa,
    implica generar un procedimiento para estimar la actividad del vector
    y eventualmente independizarse de las mediciones de campo. Esto dado
    que realizar la medición de oviposición en 50 casas todas las semanas, durante
    largos períodos de tiempo (como utilizamos para generar los modelos) tiene
    un costo extremádamente alto.


  \par Este estudio resulta ser un avance sobre trabajos previos en el área
    de la epidemiología, que consideran modelos estadíticos utilizando
    relaciones lineales \cite{models_predicting, modis_data, ndwi_erffectiveness}.
    Estas mejorías fueron obtenidas utilizando herramientas de Aprendizaje
    Automático que, en este caso, no requieren de un esfuerzo adicional de
    parte del usuario.

  \par La propuesta muestra que algunas herramientas \textit{out-of-the-shelf}
    son capaces de manejar las complejas relaciones entre variables, proporcionando
    así una forma de abordar este importante problema. Este enfoque interdisciplinario
    proporciona nuevas herramientas para los profesionales que se encuentran
    trabajando en esta área.

  \par A su vez, este trabajo es un ejemplo de cómo el uso de herramientas
    automáticas para la configuración de algoritmos, como \textit{iRace} pueden
    reducir la complejidad del ajuste de hiperparámetros y
    proveer un marco de referencia para la selección de modelos.
    Adicionalmente, mostramos la importancia de la utilización de la Validación
    Cruzada (VC), raramente utilizada en usuarios del Sensado Remoto.
    Utilizamos VC para disminuir la dependencia de los resultados de evaluación
    sobre una selección particular de los conjuntos de entrenamiento y validacíon
    en la etapa de elección del modelo. Utilizamos
    un procedimiento particular de CV para series de tiempo. Todos los modelos
    aquí discutidos pueden ser ejecutados con \textit{scripts} de Python
    disponibles libres en \url{https://github.com/juansca/modeling-mosquitos}.

  \par Encontramos que la Regresión por K-Vecinos Cercanos (KNNR), el
    Perceptron Multicapa (MLP) y la \textit{Support Vector Machine} mejoran
    considerablemente los modelos predictivos de la población de vectores utilizando
    variables ambientales extraídas de información satelital como \textit{features}.
    El desempeño de estos algoritmos puede ser mejorado sustancialmente utilizando
    conjuntos de datos más grandes, o bien realizando ajustes más finos en los modelos
    y otras técnicas de mayor complejidad dentro del área del Aprendizaje Automático.
    A pesar de que el período utilizado es largo en comparación con trabajos similares
    sobre poblaciones del vector.

\end{document}
