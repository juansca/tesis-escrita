\documentclass[12pt,spanish,fleqn,openany,letterpaper,pagesize]{scrbook}

\usepackage[utf8]{inputenc}
\usepackage[spanish]{babel}
\usepackage{fancyhdr}
\usepackage{epsfig}
\usepackage{epic}
\usepackage{eepic}
\usepackage{amsmath}
\usepackage{threeparttable}
\usepackage{amscd}
\usepackage{here}
\usepackage{graphicx}
\usepackage{lscape}
\usepackage{tabularx}
\usepackage{subfigure}
\usepackage{longtable}


\usepackage{rotating} %Para rotar texto, objetos y tablas seite. No se ve en DVI solo en PS. Seite 328 Hundebuch
                        %se usa junto con \rotate, \sidewidestable ....


\renewcommand{\theequation}{\thechapter-\arabic{equation}}
\renewcommand{\thefigure}{\textbf{\thechapter-\arabic{figure}}}
\renewcommand{\thetable}{\textbf{\thechapter-\arabic{table}}}


\pagestyle{fancyplain}%\addtolength{\headwidth}{\marginparwidth}
\textheight22.5cm \topmargin0cm \textwidth16.5cm
\oddsidemargin0.5cm \evensidemargin-0.5cm%
\renewcommand{\chaptermark}[1]{\markboth{\thechapter\; #1}{}}
\renewcommand{\sectionmark}[1]{\markright{\thesection\; #1}}
\lhead[\fancyplain{}{\thepage}]{\fancyplain{}{\rightmark}}
\rhead[\fancyplain{}{\leftmark}]{\fancyplain{}{\thepage}}
\fancyfoot{}
\thispagestyle{fancy}%


\addtolength{\headwidth}{0cm}
\unitlength1mm %Define la unidad LE para Figuras
\mathindent0cm %Define la distancia de las formulas al texto,  fleqn las descentra
\marginparwidth0cm
\parindent0cm %Define la distancia de la primera linea de un parrafo a la margen

%Para tablas,  redefine el backschlash en tablas donde se define la posici\'{o}n del texto en las
%casillas (con \centering \raggedright o \raggedleft)
\newcommand{\PreserveBackslash}[1]{\let\temp=\\#1\let\\=\temp}
\let\PBS=\PreserveBackslash

%Espacio entre lineas
\renewcommand{\baselinestretch}{1.1}

%Neuer Befehl f\"{u}r die Tabelle Eigenschaften der Aktivkohlen
\newcommand{\arr}[1]{\raisebox{1.5ex}[0cm][0cm]{#1}}

%Neue Kommandos
\usepackage{Befehle}


%Trennungsliste
\hyphenation {Reaktor-ab-me-ssun-gen Gas-zu-sa-mmen-set-zung
Raum-gesch-win-dig-keit Durch-fluss Stick-stoff-gemisch
Ad-sorp-tions-tem-pe-ra-tur Klein-schmidt
Kohlen-stoff-Mole-kular-siebe Py-rolysat-aus-beu-te
Trans-port-vor-gan-ge}


\begin{document}

\justifying

\chapter{Discusión y Conclusiones}

  \par Dengue, Chikungunya y Zika son enfermedades virales para las cuales no
    existen, al día de hoy, vacunas de prevención. Por lo tanto, el control
    más efectivo proviene de prevenir la propagación del mosquito
    \textit{Aedes Aegipty} (\textit{Linneaus}) y, por tanto, saber sobre la
    dinámica de su población es de suma importancia.

  \par Este trabajo, por un lado, presenta un \textit{framework} de simple
    utilización para el
    pronóstico de la oviposición utilizando
    únicamente variables ambientales extraídas de información satelital y
    herramientas de Aprendizaje Automático de libre acceso. Estas herramientas
    son una mejora al sistema operacional de riesgo de
    Argentina \cite{porcasi_operative}. A su vez, por la arquitectura del mismo,
    es posible agregar modelos nuevos y modificar las variables independientes a utilizar como
    predictores (\textit{features}) de una manera sencilla.

  \par En este caso, utilizamos variables ambientales derivadas de información satelital
    (temperatura, humedad y precipitación) operacionalmente disponibles para
    construir modelos temporales capaces de predecir la actividad de oviposición
    fuera de las casas. En ese sentido, nuestra perspectiva, completamente operativa,
    implica generar un procedimiento para estimar la actividad del vector
    y eventualmente independizarse de las mediciones de campo. Dicha contribución
    se considera de alto valor, entre otras cosas, porque realizar la medición
    de oviposición en 50 casas todas las semanas, durante
    largos períodos de tiempo (como utilizamos para generar los modelos) tiene
    un costo extremádamente alto.

  \par Este estudio resulta ser un avance sobre trabajos previos en el área
    de la epidemiología panorámica, donde se consideran modelos estadíticos utilizando
    relaciones lineales \cite{models_predicting, modis_data, ndwi_erffectiveness}
    en terminos de la capacidad predictiva de los modelos desarrollados aqui.
    Estas mejoras fueron obtenidas utilizando herramientas de Aprendizaje
    Automático que, en este caso, no requieren de un esfuerzo adicional de
    parte del usuario.

  \par La metodologia implementada muestra que algunas herramientas \textit{out-of-the-shelf}
    son capaces de manejar las complejas relaciones entre variables, proporcionando
    así una forma de abordar este importante problema. Este enfoque interdisciplinario
    proporciona nuevas herramientas para los profesionales que se encuentran
    trabajando en esta área.

  \par A su vez, este trabajo es un ejemplo de cómo el uso de herramientas
    automáticas para la configuración de algoritmos, como \textit{iRace} pueden
    reducir la complejidad del ajuste de hiperparámetros de los modelos y
    proveer un marco de referencia para la selección de los mismos.
    Adicionalmente, se mostra la importancia de la utilización de la Validación
    Cruzada (VC), raramente utilizada en usuarios del Sensado Remoto.
    Utilizamos VC para disminuir la dependencia de los resultados de evaluación
    sobre una selección particular de los conjuntos de entrenamiento y validacíon
    en la etapa de elección del modelo. Aqui se Utiliza
    un procedimiento particular de CV para series de tiempo. Todos los modelos
    aquí discutidos pueden ser ejecutados con \textit{scripts} de Python
    disponibles libres en \url{https://github.com/juansca/modeling-mosquitos}.

  \par En lo que respecta a la comparacion de algoritmos, se encontro que la Regresión
  por K-Vecinos Cercanos (KNNR), el
    Perceptron Multicapa (MLP) y la \textit{Support Vector Machine} resultan ser los
     modelos predictivos de la población de vectores utilizando
    variables ambientales extraídas de información satelital como \textit{features}
    que mejores resultados arrojan.


    A pesar de que el período utilizado es largo en comparación con trabajos similares
    sobre poblaciones del vector, el desempeño de estos algoritmos puede ser
    mejorado sustancialmente utilizando
    conjuntos de datos más grandes. Otra manera de mejorarlos seria o bien
    realizar ajustes más finos en el modelado
    y/o utilizar otras técnicas de mayor complejidad dentro del área del Aprendizaje Automático.

  \par La otra importante contribución de este trabajo está relacionada con la
    necesidad de poseer modelos de oviposición para distintas ciudades, evitando
    el gran costo de la recolección de datos y el entrenamiento para cada uno de
    las ciudades o puntos para los cuales se quiera poseer datos. Aquí presentamos
    una forma de establecer relaciones entre los distintos lugares geográficos teniendo
    en cuenta las características ambientales que poseen. La hipótesis más fuerte que
    asumimos es la que nos dice que el comportamiento de los vectores está altamente
    correlacionado (al menos dentro de cierto rango) a las características ambientales
    del punto en el que que se observa.

  \par Asi se presenta, desarrolla e implementa el concepto de Distancia Ambiental
    Normalizada, el cual permite
    llevar a cabo lo mencionado en el párrafo anterior estableciendo una distancia
    vectorial utilizando el espacio de características ambientales extraídas de
    información satelital, en vez del espacio geográfico.

  \par En conjunto, ambas contribuciones aportan un muy alto valor de capacidad de
    mejora al sistema operacional de riesgo de la república Argentina. A su vez,
    en perspectiva, aporta valor a la proyección de mejora de dichos modelos por
    su facilidad de uso y extrapolación a distintas zonas.

  \par Otro punto de valor del trabajo es el caracter integrador e
    interdisciplinario del mismo, demostrando la utilidad y la necesidad
    de la incersión del Aprendizaje Automático en áreas de impacto social.

  \par A su vez, cabe destacar que lo desarrollado involucra conocimientos de
  diversas áreas de las Ciencias de la Computación abarcando temáticas, por ejemplo, de
  Ingeniería del Software, a la hora de realizar el análisis de requerimientos,
  generación de la arquitectura y establecer la metodología de trabajo.
  Por otra parte, también se utilizan conocimientos de estadística, modelos y
  simulación y distintas áreas de matemática para el entendimiento de los
  distintos algoritmos, tomar decisiones con respecto a
  ellos y a las hipótesis y conclusiones. Muchos otros conceptos aprendidos
  a nivel general por las distintas materias han sido aplicados en el desarrollo.
  Es por esto que me resulta de suma importancia mencionar que lo realizado en
  este trabajo, con las características interdisciplinarias y la envergadura
  del mismo, me permitió integrar, de una manera muy contructiva para mi
  desarrollo profesional, todo lo aprendido y adquirido a lo largo de la carrera.

  \par Es importante rezaltar finalmente  que los resultados y metodologías incluidos en este
    trabajo de grado han dado lugar a tres publicaciones indexadas en la base de
    datos scopus, a saber:

    \begin{itemize}
      \item pooooner las ciiitas
    \end{itemize}
    \textit{Modeling Dengue Vector Population Using Remotely Sensed Data and
    Machine Learning} \cite{scavuzzo2018modeling} y
    \textit{Generalización espacial de modelos epidemiológicos basada en el
    concepto de Distancia Ambiental Normalizada NED} en
    en la revista \textit{Acta Tropica}
    de \textit{Elsevier} (\url{https://www.journals.elsevier.com/acta-tropica})
    y \textit{IEEE Xplore Digital Library} (\url{https://ieeexplore.ieee.org/})
    correspondientemente.
\end{document}
